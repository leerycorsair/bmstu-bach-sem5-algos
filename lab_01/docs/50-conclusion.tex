\chapter*{Заключение}
\addcontentsline{toc}{chapter}{Заключение}

В ходе выполнения работы были выполнены все поставленные задачи и изучены методы динамического программирования на основе алгоритмов вычисления расстояния Левенштейна.

Экспериментально были установлены различия в производительности различных алгоритмов вычисления расстояния Левенштейна. Рекурсивный алгоритм Левенштейна работает на порядок дольше  реализации с кэшированием, время его работы увеличивается в геометрической прогрессии. На словах длиной 12 символов, реализация алгоритма Левенштейна с кэшированием превосходит по времени работы рекурсивную в ~200 000 раз. Алгоритм Дамерау — Левенштейна по времени выполнения сопоставим с алгоритмом Левенштейна. В нём добавлены дополнительные проверки, и по сути он является алгоритмом другого смыслового уровня. При вычислении расстояния Дамерау--Левенштейна использование итеративного подхода дает выигрышь по времени в ~400 000 раз.

Теоретически было рассчитано использования памяти в каждом из алгоритмов вычисления расстояния Левенштейна. Обычные матричные алгоритмы потребляют намного больше памяти, чем рекурсивные, за счет дополнительного выделения памяти под матрицы и большее количество локальных переменных, однако при длинных строках глубина рекурсии становится слишком большой и рекурсивные алгоритмы без кэширования начинают проигрывать и по памяти.