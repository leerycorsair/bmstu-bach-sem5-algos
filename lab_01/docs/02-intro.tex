\chapter*{Введение}
\addcontentsline{toc}{chapter}{Введение}

Современные компьютеры являются мощными устройствами для
работы с текстом, они осуществляют его хранение, передачу и
обработку. Согласно статистике каждую минуту происходит создание
порядка 16 миллионов текстовых сообщений, больше половины
человечества используют электронную почту, отправляя ежедневно 267 миллиардов электронных писем, и число пользователей только
увеличивается, так к 2023 году ожидается их рост до 5.3 миллиарда. При этом почти 60 \% электронных посланий содержат опечатки или написаны с ошибками.

Автоматическая проверка орфографии — одна из актуальных
проблем в области обработки естественного языка, универсального
решения для нее до сих пор не представлено, однако на протяжении
всей своей истории коррекция орфографии являлась актуальной задачей прикладной лингвистики. Таким образом, существует задача определения максимально похожего слова к написанному, которую в свою очередь описывает расстояние Левенштейна.

Данная метрика и ее вариации активно и часто используются: 
\begin{enumerate}[label={\arabic*)}]
	\item для исправления ошибок в слове (в поисковых системах, базах данных, при вводе текста, при автоматическом распознавании отсканированного текста или речи);
	\item в биоинформатике для сравнения генов, хромосом и белков.
\end{enumerate}

Расстояние Левенштейна \cite{Levenshtein} — метрика, измеряющая разность между двумя последовательностями символов. Она определяется как минимальное количество односимвольных операций (а именно вставки, удаления, замены), необходимых для превращения одной последовательности символов в другую. В общем случае, операциям, используемым в этом преобразовании, можно назначить разные цены.

Расстояние Дамерау — Левенштейна яляется модификацией расстояния Левенштейна: к операциям вставки, удаления и замены символов, определённых в расстоянии Левенштейна добавлена операция транспозиции (перестановки) символов.



Целью данной лабораторной работы являются изучение и реализация алгоритмов Левенштейна и Дамерау--Левенштейна.

Для достижения указанной выше цели следует выполнить следующие задачи:
\begin{itemize}
	\item изучение алгоритмов Левенштейна и Дамерау--Левенштейна;
	\item привести схемы указанных алгоритмов поиска редакционного расстояния;
	\item применение методов динамического программирования для реализации указанных алгоритмов;
	\item выполнение сравнительного анализа линейной и рекурсивной реализаций алгоритмов по затрачиваемым ресурсам (по памяти и по времени);
	\item описание и обоснование полученных результатов в отчете о выполненной лабораторной работе.
\end{itemize}
