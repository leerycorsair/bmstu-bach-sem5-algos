\chapter{Исследовательская часть}

\section{Пример работы}

Демонстрация работы программы приведена на рисунке \ref{img:demo}.

\boximg{160mm}{demo}{Демонстрация работы алгоритмов нахождения расстояния Левенштейна и Дамерау -- Левенштейна}

\section{Технические характеристики}

Технические характеристики устройства, на котором выполнялось тестирование:

\begin{itemize}
	\item Операционная система: Windows 10 64-bit \cite{windows}.
	\item Память: 16 GB.
	\item Процессор: AMD Ryzen 5 4600H \cite{amd} @ 3.00 GHz
\end{itemize}

Тестирование проводилось на ноутбуке при включённом режиме производительности. Во время тестирования ноутбук был нагружен только системными процессами.

\section{Время выполнения алгоритмов}

Алгоритмы тестировались при помощи написания <<бенчмарков>> \cite{gotest}, предоставляемых встроенными в Go средствами. Для такого рода тестирования не нужно самостоятельно указывать количество повторов. В библиотеке для тестирования существует константа $N$, которая динамически варьируется в зависимости от того, был ли получен стабильный результат или нет.

Результаты замеров приведены в таблице \ref{tbl:time}. В данной таблице для значений, для которых тестирование не выполнялось, в поле результата находится NaN.
На рисунках \ref{plt:time_levenshtein} и \ref{plt:time_dl} приведены графики зависимостей времени работы алгоритмов от длины строк.
\captionsetup{singlelinecheck = false, justification=raggedright}
\begin{table}[h]
	\begin{center}
		\caption{Замер времени для строк, размером от 5 до 100}
		\label{tbl:time}
		\begin{tabular}{|c|c|c|c|c|}
			\hline
			                      & \multicolumn{4}{c|}{\bfseries Время, нс}                                    \\ \cline{2-5}
			\bfseries Длина строк & \bfseries LevRec & \bfseries LevCached & \bfseries DamLevRec & \bfseries DamLevMtr
			\csvreader{inc/csv/time.csv}{}
			{\\\hline \csvcoli&\csvcolii&\csvcoliii&\csvcoliv&\csvcolv}
			\\\hline
		\end{tabular}
	\end{center}
\end{table}

\begin{figure}[h]
	\centering
	\begin{tikzpicture}
		\begin{axis}[
			axis lines=left,
			xlabel=Длина строк,
			ylabel={Время, нс},
			legend pos=north west,
			ymajorgrids=true
		]
			\addplot table[x=len,y=time,col sep=comma] {inc/csv/time_cached.csv};
			\addplot table[x=len,y=time,col sep=comma] {inc/csv/time_rec.csv};
			\legend{С кэшированием, рекурсивный}
		\end{axis}
	\end{tikzpicture}
	\captionsetup{justification=centering}
	\caption{Зависимость процессорного времени работы алгоритма вычисления расстояния Левенштейна от длины строк (рекурсивная и с кэшированием реализации)}
	\label{plt:time_levenshtein}
\end{figure}

\begin{figure}[h]
	\centering
	\begin{tikzpicture}
		\begin{axis}[
			axis lines=left,
			xlabel=Длина строк,
			ylabel={Время, нс},
			legend pos=north west,
			ymajorgrids=true
		]
			\addplot table[x=len,y=time,col sep=comma] {inc/csv/time_lr.csv};
			\addplot table[x=len,y=time,col sep=comma] {inc/csv/time_dl.csv};
			\legend{Левенштейн, Д. — Левенштейн}
		\end{axis}
	\end{tikzpicture}
	\captionsetup{justification=centering}
	\caption{Зависимость процессорного времени работы реализаций алгоритмов Левенштейна и Дамерау — Левенштейна(с кэшированием и итеративной реализациями)}
	\label{plt:time_dl}
\end{figure}

\captionsetup{singlelinecheck = false, justification=centering}

\section{Использование памяти}

Алгоритмы Левенштейна и Дамерау — Левенштейна не отличаются друг от друга с точки зрения использования памяти, следовательно, достаточно рассмотреть лишь разницу рекурсивной и матричной реализаций этих алгоритмов.

Максимальная глубина стека вызовов при рекурсивной реализации равна сумме длин входящих строк, соответственно, максимальный расход памяти (\ref{for:99})
\begin{equation}
(\mathcal{C}(S_1) + \mathcal{C}(S_2)) \cdot (2 \cdot \mathcal{C}\mathrm{(string)} + 3 \cdot \mathcal{C}\mathrm{(int)}),
\label{for:99}
\end{equation}
где $\mathcal{C}$ — оператор вычисления размера, $S_1$, $S_2$ — строки, $\mathrm{int}$ — целочисленный тип, $\mathrm{string}$ — строковый тип.

Использование памяти при итеративной реализации теоретически равно
\begin{equation}
(\mathcal{C}(S_1) + 1) \cdot (\mathcal{C}(S_2) + 1) \cdot \mathcal{C}\mathrm{(int)} + 10\cdot \mathcal{C}\mathrm{(int)} + 2 \cdot \mathcal{C}\mathrm{(string)}.
\end{equation}

Выделение памяти при работе алгоритмов указано на рисунке \ref{img:memory}.

\boximg{120mm}{memory}{Замеры производительности алгоритмов, выполненные при помощи команды \code{go test -bench . -benchmem}}

\section{Вывод}

Рекурсивный алгоритм Левенштейна работает на порядок дольше  реализации с кэшированием, время его работы увеличивается в геометрической прогрессии. На словах длиной 12 символов, реализация алгоритма Левенштейна с кэшированием превосходит по времени работы рекурсивную в ~200 000 раз. Алгоритм Дамерау — Левенштейна по времени выполнения сопоставим с алгоритмом Левенштейна. В нём добавлены дополнительные проверки, и по сути он является алгоритмом другого смыслового уровня. При вычислении расстояния Дамерау--Левенштейна использование итеративного подхода дает выигрышь по времени в ~400 000 раз.

Но по расходу памяти итеративные алгоритмы проигрывают рекурсивному: максимальный размер используемой памяти в них растёт как произведение длин строк, в то время как у рекурсивного алгоритма — как сумма длин строк.
