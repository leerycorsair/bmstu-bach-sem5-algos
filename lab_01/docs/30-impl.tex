\chapter{Технологическая часть}

В данном разделе приведены требования к программному обеспечению, средства реализации и листинги кода.

\section{Требования к ПО}

К программе предъявляется ряд требований:
\begin{itemize}
	\item на вход подаются две строки в английской или русской раскладе, в том числе содержащие цифры, специальные символы или пустые;
	\item на выходе — искомое расстояние для всех четырех методов и матрицы расстояний для соотвествующих алгоритмов.
\end{itemize}

\section{Средства реализации}

В качестве языка программирования для реализации данной лабораторной работы был выбран многопоточный язык GO \cite{golang}. Данный выбор обусловлен моим желанием расширить свои знания в области применения данного язкыа. Так же данный язык предоставляет средства тестирования разработанного ПО.

\section{Листинг кода}

В листингах \ref{lst:levenshtein_rec}--\ref{lst:damlev_mtr} приведены реализации алгоритмов Левенштейна и Дамерау — Левенштейна, а также вспомогательные функции.

\clearpage

\captionsetup{singlelinecheck = false, justification=raggedright}

\begin{lstinputlisting}[
	caption={Алгоритм Левенштейна},
	label={lst:levenshtein_rec},
	style={go},
	linerange={3-31},
	]{../src/levenshtein/levenshtein.go}
\end{lstinputlisting}

\clearpage


\begin{lstinputlisting}[
	caption={Алгоритм Левенштейна с кэшированием},
	label={lst:levenshtein_cached},
	style={go},
	linerange={33-66},
	]{../src/levenshtein/levenshtein.go}
\end{lstinputlisting}

\clearpage


\begin{lstinputlisting}[
	caption={Алгоритм Дамерау--Левенштейна},
	label={lst:damlev},
	style={go},
	linerange={140-177},
	]{../src/levenshtein/levenshtein.go}
\end{lstinputlisting}

\clearpage

\begin{lstinputlisting}[
	caption={Алгоритм Дамерау--Левенштейна c матрицей},
	label={lst:damlev_mtr},
	style={go},
	linerange={92-138},
	]{../src/levenshtein/levenshtein.go}
\end{lstinputlisting}
\clearpage



В таблице \ref{tabular:functional_test} приведены функциональные тесты для алгоритмов вычисления расстояния Левенштейна и Дамерау — Левенштейна. Все тесты пройдены успешно (таблица \ref{tabular:functional_test_res}).


\begin{table}[h]
	\begin{center}
		\caption{\label{tabular:functional_test} Функциональные тесты}
		\begin{tabular}{|c|c|c|c|}
			\hline
			                    &                    & \multicolumn{2}{c|}{\bfseries Ожидаемый результат}    \\ \cline{3-4}
			\bfseries Строка 1  & \bfseries Строка 2 & \bfseries Левенштейн & \bfseries Дамерау — Левенштейн
			\csvreader{inc/csv/functional-test.csv}{}
			{\\\hline \csvcoli&\csvcolii&\csvcoliii&\csvcoliv}
			\\\hline
		\end{tabular}
	\end{center}
\end{table}

\begin{table}[h]
	\begin{center}
		\caption{\label{tabular:functional_test_res} Результат работы программы}
		\begin{tabular}{|c|c|c|c|}
			\hline
			&                    & \multicolumn{2}{c|}{\bfseries Фактический результат}    \\ \cline{3-4}
			\bfseries Строка 1  & \bfseries Строка 2 & \bfseries Левенштейн & \bfseries Дамерау — Левенштейн
			\csvreader{inc/csv/functional-test.csv}{}
			{\\\hline \csvcoli&\csvcolii&\csvcoliii&\csvcoliv}
			\\\hline
		\end{tabular}
	\end{center}
\end{table}
\captionsetup{singlelinecheck = false, justification=centering}

\section{Вывод}

Были разработаны и протестированы спроектированные алгоритмы: вычисления расстояния Левенштейна рекурсивно,рекурсивно с кэшированием, а также вычисления расстояния Дамерау — Левенштейна рекурсивно и с заполнением матрицы.
