\chapter{Аналитическая часть}

В данном разделе выполняется формальная постановка задачи коммивояжера и математическое описание каждого из алгоритмов.

\section {Постановка задачи}

Пусть есть \textit{N} городов, соединенных между собой. Странствующий торговец (т.н. коммивояжер) объезжает эти города. Ему необходимо составить такой маршрут, чтобы он был наикратчайшим, и при этом каждый из N городов должен быть посещен ровно по одному разу. В конце маршрута коммивояжер должен вернуться в город, с которого он начинал маршрут.

Для возможности применения математического аппарата для решения задачи её следует представить в виде математической модели.

\img{50mm}{graph}{Симметричная задача для четырех городов}

Задачу коммивояжера можно представить в виде модели на графе.
Вершины графа - города, ребра между вершинами - пути между этими
городами. Каждому ребру в соответствие ставится его вес, который в данной задаче определяет расстояние между городами, определяемыми вершинами.

Другими словами, задан взвешенный полностью связный граф. Для него необходимо найти гамильтонов цикл минимального веса.

\section{Алгоритм полного перебора}
Задачу коммивояжера можно решить, рассмотрев все возможные комбинации городов. Для каждого из маршрутов высчитывается его суммарная длина. Выбирается такой маршрут, для которого его суммарная длина минимальна.

Такой алгоритм гарантирует точное решение, однако, как известно, алгоритмическая сложность подобных алгоритмов составляет O(n!), поэтому даже при небольшом количестве городов решение за оптимальное время становится невозможным.

\section{Муравьиный алгоритм}

Муравьиный алгоритм основан на особенностях поведения мура­вьёв в природе. При поиске путей к источникам пищи муравьи помечают пройденный путь специальным веществом --- феромоном.

Феромон играет роль непрямой обратной связи: чем больше муравьёв движется по помеченному пути, тем сильнее он привлекает других насекомых. По истечении некоторого времени феромон испаряется.

Таким образом, большая часть муравьёв будет передвигаться от муравейника до найденного источника пищи по одному и тому же пути.

В вершинах описанного ранее графа располагаются муравьи, которые перемещаются по рёбрам графа. Для каждого муравья вводится понятие памяти (или списка за­претов) --- списка пройденных им за день узлов. Выбирая узел для следующего шага, муравей помнит об уже пройденных узлах и не рассмат­ривает их в качестве возможных для перехода. Также вводится понятие зрения: муравей может оценить длины рёбер до следующих городов.

Пусть имеется $N$ городов и $N$ муравьёв. Вводится матрица смеж­ности $D$ и матрица феромонов $T$.

Колония помещается в одну из вершин графа, называемую стар­товой вершиной. Все рёбра графа помечаются одинаковым фоновым зна­чением феромона $T_{ij} = T_0$. Муравьи каждое утро выходят из муравейника и независимо друг от друга перемещаются между вершинами.

При выборе следующей вершины для посещения муравьи опира­ются на значения концентрации феромона на смежных рёбрах, которые не находятся в списке запрета, и на значения длин этих рёбер.
К закату муравьи возвращаются в муравейник и обсуждают, кто нашёл самый короткий маршрут. Если новый маршрут лучше прежнего, то его запоминают. Ночью происходит обновление феромонов на рёбрах.

Пусть $k$-ый муравей находится в $i$-ой вершине, а его запретный список $V_k$ ещё не до конца заполнен. Тогда вероятность перемещения в $j$-ую вершину определяется следующей формулой.

\begin{equation*}
	P_{ij}^k = \begin{cases}
		\dfrac{T_{ij}^\alpha\cdot\eta_{ij}^\beta}{\sum\limits_{r\in V_{ik}} T_{ir}^\alpha\cdot\eta_{ir}^\beta},&\text{вершина $j$ сегодня не была посещена $k$-ым муравьём}\\
		0, & \text{иначе}
	\end{cases}
\end{equation*}
\vspace{5mm}

Здесь $\alpha$ --- коэффициент жадности, $\beta$ --- коэффициент стадности, а $\eta_{ij} = \dfrac{1}{D_{ij}}$ --- привлекательность ребра. $\alpha + \beta = \text{const}$.

Если $\alpha = 0$, то алгоритм вырождается в чисто стадный (куда стадо, туда и все). Если $\beta = 0$, то алгоритм вырождается в жадный, потому что опыт колонии не учитывается.

Когда все муравьи завершили обход, происходит обновление фе­ромона по следующей формуле.

\begin{equation*}
	T_{ij}(t + 1) = (1 - p)\cdot(T_{ij}(t) + \Delta T_{ij}),\ \text{где}
\end{equation*}
%
\begin{equation*}
	\Delta T_{ij} = \sum\limits_{k = 1}^N \Delta T_{ij}^k
\end{equation*}
%
\begin{equation*}
	\Delta T_{ij}^k = \begin{cases}
		\dfrac{Q}{L_k}, & \text{если ребро посещено $k$-ым муравьём}\\
		0, & \text{иначе}
	\end{cases}
\end{equation*}
\vspace{5mm}

$L_k$ --- длина пути $k$-го муравья, $Q$ --- некоторая константа, соразмерная длине лучшего пути (запас феромона), $\Delta T_{ij}^k$ --- количество феромона, оставленного $k$-ым муравьём на $ij$-ом ребре. Таким образом, чем короче путь $k$-го муравья, тем больше феромона он оставит на пройденных рёбрах.

Описанные действия представляют собой одну итерацию муравьи­ного алгоритма. Итерации повторяются до тех пор, пока не окажется выполненным какой-либо из критериев останова алгоритма: исчерпа­но число итераций, достигнута нужная точность, получен единственный путь (алгоритм сошелся к некоторому решению).

Нужно также учитывать случай, когда феромон может обнулить­ся --- тогда обнулится вся вероятность прохода по этому ребру. Чтобы этого не допустить, вводится константа фонового значения феромона. Если значение феромона становится меньше фонового, то ему присваивается фоновое значение.

Чтобы оставить возможность прохода по не самому оптимальному с точки зрения вероятности пути, муравей подбрасывает монетку, по­лучая случайное число от $0$ до $1$. Если искать только максимум вероят­ностей --- это в своем роде жадное решение.

\section{Параметризация муравьиного алгоритма}

Для эвристических методов проводится параметризация. Параметризация заключается в определении таких параметров или настроек работы метода, при которых для выбранного класса данных задача решается с наилучшим качеством.

Грубо говоря, метод складывается из некоторого набора алгорит­мов и некоторой математической модели представления задачи в предметной области. Для эвристических методов вводятся некоторые допол­нительные понятия, которые составляют часть модели, которая пред­ставляет предметную область.

Здесь вводятся понятия муравья (некоторой части алгорит­ма, которая соответствует виденью муравья части задачи) и феромо­на (средства обмена информацией).

Для этой задачи могут быть разные классы данных. Предполо­жим, это один вид карт с однотипным разбросом длин рёбер. Нужно подстроить метод под какой-то конкретный класс данных, так как приду­мать универсальный алгоритм трудно.

У алгоритма есть 3 параметра: $\alpha$ --- коэффициент жадности, $p$ --- коэффициент испарения феромона и $T_\text{max}$ --- время жизни колонии.

Чтобы провести параметризацию, необходимо выбрать некоторое ограни­ченное количество значений исследуемых параметров, затем решить задачу при этих коэффициентах. Полученное решение сравнивается с эталон­ным значением, определяемым полным перебором. Необходи­мо отыскать набор параметров, который будет наиболее близок к эталон­ному решению. Это делается с помощью построения таблицы.

\section{Вывод}

Для ПО, решающего поставленную задачу следует выделить следующие требования: 
\begin{itemize}
	\item входными данными является ориентированный граф, заданной целочисленной матрицей смежности;
	\item введенный граф является связным (т.е. на нем разрешима задача коммивояжера);
	\item ПО имеет подсказки ввода;
	\item ПО должно обеспечить методы логирования полученных эксперементальных данных;	
	\item ПО выводит минимальную длину пути и сам путь.
\end{itemize}

В данном разделе были рассмотрены два алгоритма решения задачи коммивояжера: алгоритм полного перебора и муравьиный алгоритм. Для каждого из алгоритмов были выделены основные идеи решения поставленной задачи, а также определена асимптотическая сложность.


