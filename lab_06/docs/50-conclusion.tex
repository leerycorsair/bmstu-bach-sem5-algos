\chapter*{Заключение}
\addcontentsline{toc}{chapter}{Заключение}

В ходе выполнения работы были выполнены все поставленные задачи и изучены методы динамического программирования на основе алгоритмов сортировки массивов. Был изучен и реализован эвристи­ческий метод решения задачи коммивояжёра --- муравьиный алгоритм. Его идея основывается на биологическом принципе поведения муравьёв.

Большинство эвристических методов, к которым относится мура­вьиный алгоритм, не гарантируют точное решение, но являются доста­точными для решения задачи за оптимальное время.

Эвристические методы решают задачу не на универсальном наборе данных, а только на заранее заданном единообразном типе матриц. По этой причине важным этапом реализации эвристических методов является его параметризация --- настройка на сооветствующий тип данных.

Параметризация заключается в варьировании параметров, реше­нии задачи с этими параметрами и последующем сравнении получен­ного значения с эталонным, полученным точным, но долгим методом. В данной лабораторной работе параметризация настраивалась на трёх картах, таким образом суммарная погрешность относительно эталонного решения определялась как максимум из погрешностей для каждой из карт. Выбирая оптимальные настройки, нужно добиться того, чтобы худшая погрешность среди трёх карт была минимальной.

С помощью реализации муравьиного алгоритма удалось добиться как минимум десятикратного выигрыша на небольших размерностях. Чем больше размерность матрицы смежности, тем больше выигрыш.


