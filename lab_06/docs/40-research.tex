\chapter{Исследовательская часть}

\section{Пример работы}

Демонстрация работы программы приведена на рисунке \ref{img:demo}.

\boximg{60mm}{demo}{Демонстрация работы алгоритмов}

\section{Технические характеристики}

Технические характеристики устройства, на котором выполнялось тестирование:

\begin{itemize}
	\item Операционная система: Windows 10 64-bit \cite{windows}.
	\item Память: 16 GB.
	\item Процессор: AMD Ryzen 5 4600H \cite{amd} @ 3.00 GHz.
\end{itemize}

Тестирование проводилось на ноутбуке при включённом режиме производительности. Во время тестирования ноутбук был нагружен только системными процессами.

\section{Время выполнения алгоритмов}

Алгоритмы тестировались при помощи написания <<бенчмарков>> \cite{gotest}, предоставляемых встроенными в Go средствами. Для такого  тестирования не нужно самостоятельно указывать количество повторов. В библиотеке для тестирования существует константа $N$, которая динамически для получения стабильного результата. На рисунке \ref{plt:time} приведен график зависимости времени работы алгоритмов для различных наборов данных.

\begin{figure}[!h]
	\centering
	\begin{tikzpicture}
		\begin{axis}[
			axis lines=left,
			xlabel=Размер массива,
			ylabel={Время, нс},
			legend pos=north west,
			ymajorgrids=true
			]
			\addplot table[x=size,y=time,col sep=comma] {inc/csv/brute.csv};
			\addplot table[x=size,y=time,col sep=comma] {inc/csv/ant.csv};
			\legend{Полный перебор, Муравьиный алгоритм}
		\end{axis}
	\end{tikzpicture}
	\captionsetup{justification=centering}
	\caption{Сравнение алгоритмов на случайных матрицах смежности.}
	\label{plt:time}
\end{figure}



\captionsetup{singlelinecheck = false, justification=raggedright}

\section{Результаты эксперимента}

В таблицах \ref{tabular:exp1} - \ref{tabular:exp2} представлены результаты эксперимента.

\begin{table}[h!]
	\begin{center}
		\caption{\label{tabular:exp1}Результат эксперимента}
		\begin{tabular}{|c|c|c|c|}
			\hline
			$\alpha$ & $\rho$ & Кол-во дней & Погрешность 	\\\hline
1 &   0.2 &     9 &    18 \\ \hline 
1 &   0.2 &    10 &    17 \\ \hline 
1 &   0.2 &    11 &    10 \\ \hline 
1 &   0.4 &     9 &    17 \\ \hline 
1 &   0.4 &    10 &     0 \\ \hline 
1 &   0.4 &    11 &     0 \\ \hline  
		\end{tabular}
	\end{center}
\end{table}

\begin{table}[h!]
	\begin{center}
		\caption{\label{tabular:exp2}Результат эксперимента (продолжение)}
		\begin{tabular}{|c|c|c|c|}
			\hline
			$\alpha$ & $\rho$ & Кол-во дней & Погрешность 	\\\hline
    1 &   0.6 &     9 &    18 \\ \hline 
1 &   0.6 &    10 &     3 \\ \hline 
1 &   0.6 &    11 &     6 \\ \hline 
1 &   0.8 &     9 &    18 \\ \hline 
1 &   0.8 &    10 &    10 \\ \hline 
1 &   0.8 &    11 &     6 \\ \hline 
1 &     1 &     9 &     0 \\ \hline 
1 &     1 &    10 &     0 \\ \hline 
1 &     1 &    11 &     0 \\ \hline 
2 &   0.2 &     9 &     3 \\ \hline 
2 &   0.2 &    10 &     3 \\ \hline 
2 &   0.2 &    11 &     3 \\ \hline 
2 &   0.4 &     9 &     6 \\ \hline 
2 &   0.4 &    10 &     0 \\ \hline 
2 &   0.4 &    11 &     6 \\ \hline 
2 &   0.6 &     9 &     6 \\ \hline 
2 &   0.6 &    10 &     6 \\ \hline 
2 &   0.6 &    11 &     0 \\ \hline 
2 &   0.8 &     9 &     0 \\ \hline 
2 &   0.8 &    10 &     0 \\ \hline 
2 &   0.8 &    11 &     7 \\ \hline 
2 &     1 &     9 &     0 \\ \hline 
2 &     1 &    10 &     3 \\ \hline 
2 &     1 &    11 &     0 \\ \hline 
3 &   0.2 &     9 &     0 \\ \hline 
3 &   0.2 &    10 &     0 \\ \hline 
3 &   0.2 &    11 &     0 \\ \hline 
3 &   0.4 &     9 &     6 \\ \hline 
3 &   0.4 &    10 &     6 \\ \hline 
3 &   0.4 &    11 &     0 \\ \hline 
3 &   0.6 &     9 &     3 \\ \hline 
3 &   0.6 &    10 &     6 \\ \hline 
3 &   0.6 &    11 &     6 \\ \hline 
3 &   0.8 &     9 &     3 \\ \hline 
3 &   0.8 &    10 &     0 \\ \hline 
3 &   0.8 &    11 &     3 \\ \hline 
3 &     1 &     9 &     0 \\ \hline 
3 &     1 &    10 &     0 \\ \hline 
3 &     1 &    11 &     0 \\ \hline 
		\end{tabular}
	\end{center}
\end{table}

\captionsetup{singlelinecheck = false, justification=centering}

\clearpage

\section{Вывод}

Было произведено сравнение количества затраченного времени полного перебора и муравьиного алгоритма.

По результатам эксперимента можно сделать вывод о том, что при небольших числах алгоритмы рабо­тают одинаково. Однако уже на размерности $10\times10$ алгоритм полного перебора начинает проигрывать, что вполне ожидаемо, так как его алгоритмическая сложность составляет $O(N!)$.

С помощью реализации муравьиного алгоритма удалось добиться как минимум десятикратного выигрыша на небольших размерностях. Чем больше будет размерность матрицы смежности, тем больше будет выиг­рыш.



