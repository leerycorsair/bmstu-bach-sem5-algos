\chapter*{Введение}
\addcontentsline{toc}{chapter}{Введение}

Задача коммивояжера - задача транспортной логистики, отрасли, занимающейся планированием транспортных перевозок. Коммивояжёру, чтобы распродать нужные и не очень нужные в хозяйстве товары, следует объехать \textit{n} пунктов и в конце концов вернуться в исходный пункт. Требуется определить наиболее выгодный маршрут объезда. В качестве меры выгодности маршрута может служить суммарное время в пути, суммарная стоимость дороги, или, в простейшем случае, длина маршрута.

Муравьиный алгоритм \cite{yualianov} - алгоритм эврестической оптимизации путем подражания муравьиной колонии, использующийся для нахождения приближенных решений задач коммивояжера, а также решения аналогичных NP-трудных задач поиска маршрутов на графах. Важно понимать, что большинство эврестических методов, к которым относится муравьиный алгоритм, не гарантируют точное решение, но являются достаточными для решения задачи за оптимальное время.

Целью данной лабораторной работы являются изучение проблемы коммивояжёра и реализация алгоритмов, решающих данную задачу: алгоритм полного перебора и муравьиный алгоритм.

Для достижения указанной выше цели следует выполнить следующие задачи:
\begin{itemize}
	\item изучить задачу коммивояжера, а также идеи ее решения с помощью муравьиного алгоритма и алгоритма полного перебора;
	\item привести схемы изученных алгоритмов;
	\item описать используемые типы данных;
	\item описать структуру разрабатываемого ПО;
	\item реализовать изученные алгоритмы;
	\item протестировать разработанное ПО;
	\item выполнить на основе эксперементадьных данных сравнительный анализ временных характеристик каждого из алгоритмов в зависимости от размерности матрицы смежности;
	\item исследовать влияние параметров муравьиного алгоритма на точность получаемого ответа;
	\item описание и обоснование полученных результатов в отчете о выполненной лабораторной работе. 
\end{itemize}

