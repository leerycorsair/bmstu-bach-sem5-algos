\chapter{Технологическая часть}

В данном разделе приведены требования к программному обеспечению, средства реализации и листинги кода.

\section{Средства реализации}

В качестве языка программирования для реализации данной лабораторной работы был выбран многопоточный язык GO \cite{golang}. Данный выбор обусловлен моим желанием расширить свои знания в области применения данного язкыа. Так же данный язык предоставляет средства тестирования разработанного ПО.

\section{Листинг кода}

В листингах \ref{lst:main}--\ref{lst:insertion} приведены реализации алгоритмов сортировок, а также вспомогательные функции.


\captionsetup{singlelinecheck = false, justification=raggedright}

\begin{lstinputlisting}[
	caption={Основной модуль программы},
	label={lst:main},
	style={go},
	linerange={8-27},
	]{../src/main.go}
\end{lstinputlisting}
\clearpage

\begin{lstinputlisting}[
	caption={Алгоритм полного перебора - Часть 1},
	label={lst:brute1},
	style={go},
	linerange={3-49},
	]{../src/ants/brute.go}
\end{lstinputlisting}
\clearpage

\begin{lstinputlisting}[
	caption={Алгоритм полного перебора - Часть 2},
	label={lst:brute2},
	style={go},
	linerange={51-64},
	]{../src/ants/brute.go}
\end{lstinputlisting}

\begin{lstinputlisting}[
	caption={Муравьиный алгоритм - Часть 1},
	label={lst:ant1},
	style={go},
	linerange={17-38},
	]{../src/ants/ant.go}
\end{lstinputlisting}

\clearpage

\begin{lstinputlisting}[
	caption={Муравьиный алгоритм - Часть 2},
	label={lst:ant2},
	style={go},
	linerange={40-84},
	]{../src/ants/ant.go}
\end{lstinputlisting}
\clearpage

\begin{lstinputlisting}[
	caption={Муравьиный алгоритм - Часть 3},
	label={lst:ant3},
	style={go},
	linerange={86-128},
	]{../src/ants/ant.go}
\end{lstinputlisting}

\clearpage

\begin{lstinputlisting}[
	caption={Муравьиный алгоритм - Часть 4},
	label={lst:ant4},
	style={go},
	linerange={130-167},
	]{../src/ants/ant.go}
\end{lstinputlisting}

\clearpage

\begin{lstinputlisting}[
	caption={Муравьиный алгоритм - Часть 5},
	label={lst:ant5},
	style={go},
	linerange={169-196},
	]{../src/ants/ant.go}
\end{lstinputlisting}

В таблице \ref{tabular:func_test} приведены функциональные тесты для алгоритмов сортировки. Все тесты пройдены успешно (таблица \ref{tabular:func_test_res}).


\begin{table}[h!]
	\begin{center}
		\caption{\label{tabular:func_test}Ожидаемый результат работы программы}
		\begin{tabular}{|c|c|c|}
			\hline
			Входная матрица смежности & Полный перебор & Муравьиный алгоритм \\ 
			\hline
			$\begin{pmatrix}
			    0 &    2 &    3 &   26 &   39 \\ 
			2 &    0 &   29 &   38 &    9 \\ 
			3 &   29 &    0 &   13 &   34 \\ 
			26 &   38 &   13 &    0 &    5 \\ 
			39 &    9 &   34 &    5 &    0 \\ 
			\end{pmatrix}$ & 32 & 32	 \\\hline
			$\begin{pmatrix}
    0 &   23 &   49 \\ 
23 &    0 &   13 \\ 
49 &   13 &    0 \\ 
			\end{pmatrix}$ & 85 & 85	\\\hline
		\end{tabular}
	\end{center}
\end{table}

\begin{table}[h!]
	\begin{center}
		\caption{\label{tabular:func_test_res}Фактический результат работы программы}
		\begin{tabular}{|c|c|c|}
			\hline
			Входная матрица смежности & Полный перебор & Муравьиный алгоритм \\ 
			\hline
			$\begin{pmatrix}
				0 &    2 &    3 &   26 &   39 \\ 
				2 &    0 &   29 &   38 &    9 \\ 
				3 &   29 &    0 &   13 &   34 \\ 
				26 &   38 &   13 &    0 &    5 \\ 
				39 &    9 &   34 &    5 &    0 \\ 
			\end{pmatrix}$ & 32 & 32	 \\\hline
			$\begin{pmatrix}
				0 &   23 &   49 \\ 
				23 &    0 &   13 \\ 
				49 &   13 &    0 \\ 
			\end{pmatrix}$ & 85 & 85	\\\hline
		\end{tabular}
	\end{center}
\end{table}


\captionsetup{singlelinecheck = false, justification=centering}

\section{Вывод}

Были разработаны и протестированы спроектированные алгоритмы: алгоритм полного перебора и муравьиный алгоритм.
