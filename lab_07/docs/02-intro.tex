\chapter*{Введение}
\addcontentsline{toc}{chapter}{Введение}

С непрерывным ростом количества доступной текстовой информации появляется потребность определенной организации ее хранения, удобного для поиска. Если текстовая информация представляет собой некоторое количество пар, то ее удобно хранить в словаре. 

Словарь(ассоциативный массив) -- это абстрактный тип данных, состоящий из коллекции элементов вида "ключ -- значение".  

Словари могут содержать достаточно большие объемы данных, поэтому задача оптимизации поиска в словаре остается актуальной \cite{yualianov}. 

Целью данной лабораторной работы являются изучение проблемы поиска записи в словаре и реализация алгоритмов, решающих данную задачу: алгоритм полного перебора, алгоритм бинарного поиска и алгоритм частотного анализа.

Для достижения указанной выше цели следует выполнить следующие задачи:
\begin{itemize}
	\item изучить задачу поиска записи в словаре, а также идеи ее решения с помощью алгоритма полного перебора, алгоритма бинарного поиска и алгоритма частотного анализа;
	\item привести схемы изученных алгоритмов;
	\item описать используемые типы данных;
	\item описать структуру разрабатываемого ПО;
	\item реализовать изученные алгоритмы;
	\item протестировать разработанное ПО;
	\item выполнить на основе эксперементадьных данных сравнительный анализ временных характеристик каждого из алгоритмов в зависимости от позиции записи в словаре;
	\item определить лучший и худщий случай для каждого из алгоритмов;
	\item описание и обоснование полученных результатов в отчете о выполненной лабораторной работе. 
\end{itemize}

