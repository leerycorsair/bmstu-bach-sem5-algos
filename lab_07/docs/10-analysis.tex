\chapter{Аналитическая часть}

В данном разделе описаны основные идеи рассматриваемых алгоритмов и выполняется формальная постановка задачи, на которой будут исследоваться данные алгоритмы.

\section{Постановка задачи}

В данной лабораторной работе поиск в словаре реализовывается на примере словаря, основанного на тексте песен британской рок-группы Bring Me The Horizon. Ключом является непосредственно слово, значением - количество раз, сколько данное слово встречается в текстах песен. Знаки препинания и прочие обозначения в тексте игнорируются.

\section{Алгоритм полного перебора}

Алгоритм полного перебора для любой задачи часто является самым примитивным и самым трудоемким алгоритмом, и в рамках рассматриваемой задачи этот случай не становится исключением. 

Для поставленной задачи алгоритм полного перебора заключается в последовательном проходе по словарю до тех пор, пока не будет найден требуемый ключ. Очевидно, что худшим случаем является ситу­ация, когда необходимый ключ находится в конце словаря либо когда этот ключ вовсе не представлен в словаре.

Трудоемкость алгоритма зависит от положения ключа в словаре - чем дальше он от начала словаря, тем больше единиц процессорного времени потребуется на поиск. Средняя трудоёмкость может быть рассчитана как математиче­ское ожидание по формуле \ref{for:simple}, где $\Omega$ – множество всех возможных случаев, k - кол-во единиц процессорного времени, затрачиваемого на одну операцию сравнения в словаре.

\begin{equation}
	\label{for:simple}
	\begin{aligned}
		\sum\limits_{i \in \Omega} p_i \cdot f_i = & k \cdot \frac{1}{N + 1} + 2 \cdot k \cdot \frac{1}{N+1} + 3 \cdot k \cdot \frac{1}{N + 1} + \cdots + N \cdot k \cdot \frac{1}{N + 1} = \\
		& = k \cdot \frac{1 + 2 + 3 + \cdots + N}{N + 1} = \frac{k \cdot N}{2}
	\end{aligned}
\end{equation}

\section{Бинарный поиск}

Алгоритм бинарного поиска применяется к заранее остортирован­
ному набору значений. В рамках потсалвенной задачи словарь должен быть остортирован по ключам по возрастанию. Основная идея бинарного поиска для поставленной задачи заклю­чается в следующем:
\begin{itemize}
	\item определение значения ключа в середине словаря. Полученное зна­чение сравнивается с искомым ключом;
	\item если ключ меньше значения середины, то поиск осуществляется
	в первой половине элементов, иначе — во второй;
	\item поиск сводится к тому, что вновь определяется значение середин­ного элемента в выбранной половине и сравнивается с ключом;
	\item процесс продолжается до тех пор, пока не будет найден элемент со значением ключа или не станет пустым интервал для поиска.
\end{itemize}
Как известно, сложность алгоритма бинарного поиска на зарнее
упорядоченном наборе данных составляет $O(log_2(n))$, где n - размер словаря.
Однако, может возникнуть ситуация, что затраты на сортировку
данных будут нивелировать преимущество быстрого поиска при больших размерностях массивов данных.

\section{Алгоритм с сегментированием словаря}

Суть алгоритма за­ключается в разбиении исходного множества пар (ключ;значение) на некоторые т.н. сегменты по заранее заданному общему признаку.
Затем сегменты сортируются по их размеру, таким образом, сег­
мент, который содержит в себе больше всего вхождений, будет распо­ложен первым, т.е. доступ к нему будет осуществляться быстрее, затем второй и так далее по убыванию.

Таким образом, на скорость доступа к сегменту влияет частота
вхождений пар значений из словаря. В каждом сегменте пары (ключ; значение) сортируются по воз­растанию ключей для дальнейшего применения бинарного поиска в кон­кретном сегменте.


Средняя трудоёмкость при множестве всех возможных случаев $\Omega$ может быть рассчитана по формуле \ref{for:segment}.

\begin{equation}
	\label{for:segment}
	\begin{aligned}
		\sum_{i \in \Omega}{\left(f_{\text{выбор сегмента i-ого элемента}} + f_{\text{бинарный поиск i-ого элемента}}\right)} \cdot p_i
	\end{aligned}
\end{equation}


\section{Вывод}

Для ПО, решающего поставленную задачу следует выделить следующие требования: 
\begin{itemize}
	\item текст песен BMTH для наполнения словаря хранится в файле bmth.txt;
	\item ПО не проверяет корректность указанных файлов;
	\item ПО имеет подсказки ввода;
	\item ПО должно обеспечить методы логирования полученных эксперементальных данных;	
	\item ПО выводит значение, хранимое под ключом в словаре, которое вводит пользователь с клавиатуры.
\end{itemize}

В данном разделе были рассмотрены три алгоритма решения задачи поиска в словаре: алгоритм полного перебора, алгоритм бинарного поиска и алгоритм частотного анализа. Для каждого из алгоритмов были выделены основные идеи решения поставленной задачи, а также определена асимптотическая сложность.


