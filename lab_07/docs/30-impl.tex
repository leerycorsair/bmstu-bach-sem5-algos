\chapter{Технологическая часть}

В данном разделе приведены требования к программному обеспечению, средства реализации и листинги кода.

\section{Средства реализации}

В качестве языка программирования для реализации данной лабораторной работы был выбран многопоточный язык GO \cite{golang}. Данный язык предоставляет средства тестирования разработанного ПО.

\section{Листинг кода}

В листингах \ref{lst:main}--\ref{lst:segment} приведены реализации алгоритмов сортировок, а также вспомогательные функции.


\captionsetup{singlelinecheck = false, justification=raggedright}

\begin{lstinputlisting}[
	caption={Основной модуль программы},
	label={lst:main},
	style={go},
	linerange={14-36},
	]{../src/main.go}
\end{lstinputlisting}
\clearpage

\begin{lstinputlisting}[
	caption={Алгоритм полного перебора},
	label={lst:brute},
	style={go},
	linerange={55-64},
	]{../src//dict/dict.go}
\end{lstinputlisting}

\begin{lstinputlisting}[
	caption={Алгоритм двоичного поиска},
	label={lst:binary},
	style={go},
	linerange={66-91},
	]{../src//dict/dict.go}
\end{lstinputlisting}

\clearpage

\begin{lstinputlisting}[
	caption={Алгоритм частотного анализа},
	label={lst:segment},
	style={go},
	linerange={93-136},
	]{../src//dict/dict.go}
\end{lstinputlisting}

\clearpage



Тестирование производилось с помощью словаря, построенном на файле, содержащем данные: "the super file yes the". В таблице \ref{tabular:func_test} приведены функциональные тесты для алгоритмов сортировки.  Все тесты пройдены успешно (таблица \ref{tabular:func_test_res}).


\begin{table}[h]
	\begin{center}
		\caption{\label{tabular:func_test} Ожидаемый результат работы программы}
		\begin{tabular}{|c|p{50mm}|c|c|c|}
			\hline
			№ & Тестовый случай & Искомый ключ & Ожидаемый результат \\ \hline
			1 & Одно вхождение &
			super &
			value = 1, exists = 1 \\ \hline
			2 & Несколько вхождений &
			the &
			value = 2, exists = 1\\ \hline
			3 & Нет вхождений &			
			no &
			value = 0, exists = 0\\ \hline
		\end{tabular}
	\end{center}
\end{table}

\begin{table}[h]
	\begin{center}
		\caption{\label{tabular:func_test_res} Фактический результат работы программы}
		\begin{tabular}{|c|p{50mm}|c|c|c|}
			\hline
			№ & Тестовый случай & Искомый ключ & Фактический результат \\ \hline
			1 & Одно вхождение &
			super &
			value = 1, exists = 1 \\ \hline
			2 & Несколько вхождений &
			the &
			value = 2, exists = 1\\ \hline
			3 & Нет вхождений &			
			no &
			value = 0, exists = 0\\ \hline
		\end{tabular}
	\end{center}
\end{table}



\captionsetup{singlelinecheck = false, justification=centering}

\section{Вывод}

Были разработаны и протестированы спроектированные алгоритмы: алгоритм полного перебора, алгоритм бинарного поиска и алгоритм частотного анализа.
