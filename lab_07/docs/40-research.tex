\chapter{Исследовательская часть}

\section{Пример работы}

Демонстрация работы программы приведена на рисунке \ref{img:demo}.

\boximg{60mm}{demo}{Демонстрация работы алгоритмов}

\section{Технические характеристики}

Технические характеристики устройства, на котором выполнялось тестирование:

\begin{itemize}
	\item Операционная система: Windows 10 64-bit \cite{windows}.
	\item Память: 16 GB.
	\item Процессор: AMD Ryzen 5 4600H \cite{amd} @ 3.00 GHz.
\end{itemize}

Тестирование проводилось на ноутбуке при включённом режиме производительности. Во время тестирования ноутбук был нагружен только системными процессами.



\section{Оценка эффективности алгоритмов}
Для выполнения лабораторной работы использовался словарь из 2130 элементов. Для оценки эффективности алгоритмов было подсчитано количество сравнений, необходимое для поиска каждого из ключей в словаре. Также были определены минимальное, среднее и максимальное количества сранений для каждого алгоритма. Для наглядности результаты были представлены в виде гистограмм.

\img{95mm}{graph_brute}{Алгоритм полного перебора, сортировка ключей по алфавиту}   
\clearpage

\img{95mm}{graph_brute}{Алгоритм полного перебора, сортировка по количеству сравнений}

\img{95mm}{graph_binary_1}{Алгоритм бинарного поиска, сортировка ключей по алфавиту}
\clearpage

\img{95mm}{graph_binary_2}{Алгоритм бинарного поиска, сортировка по количеству сравнений}

\img{95mm}{graph_segment_1}{Алгоритм частотного анализа, сортировка ключей по алфавиту}
\clearpage

\img{95mm}{graph_segment_2}{Алгоритм частотного анализа, сортировка по количеству сравнений}


\captionsetup{singlelinecheck = false, justification=raggedright}
\section{Вывод}

В данном разделе было проведено сравнение необходимых количеств сравнений при поиске ключа тремя алгоритмами: полный перебор, бинарный поиск, алгоритм частотного анализа.

\begin{table}[h]
	\begin{center}
		\caption{\label{tabular: exp} Зависимость количества сравнений от выбора алгоритма}
		\begin{tabular}{ | c | c | c | c | }
			\hline
			& Минимальное & Максимальное & Среднее \\ \hline
			Полный перебор & 1 & 2130 & 1065 \\ \hline
			Бинарный поиск & 1 & 12 & 9.54  \\ \hline
			Алгоритм частотного анализа& 1 & 8 & 4.54 \\ \hline
		\end{tabular}
	\end{center}
\end{table}

Таким образом, сегментирование словаря в среднем сокращает количество необходимых сравнений в 1.26 раз, по сравнению с бинарным поиском, и в 222.16 раз, по сравнению с полным перебором. При этом среднее количество сравнений сократилось в 2.07 раз, по сравнению с бинарным поиском, и в 266.25 раз, по сравнению с полным перебором.





