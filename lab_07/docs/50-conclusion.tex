\chapter*{Заключение}
\addcontentsline{toc}{chapter}{Заключение}

В ходе выполнения работы были выполнены все поставленные задачи и изучены методы динамического программирования на основе алгоритмов поиска в словаре. 

Сегментирование словаря в среднем сокращает количество необходимых сравнений в 1.26 раз, по сравнению с бинарным поиском, и в 222.16 раз, по сравнению с полным перебором. При этом среднее количество сравнений сократилось в 2.07 раз, по сравнению с бинарным поиском, и в 266.25 раз, по сравнению с полным перебором.

На основании проделанной работы можно сделать следующий вывод:
чем больше размер словаря, тем более рационально использовать
эффективные алгоритмы поиска. Однако стоит помнить, что бинарный
поиск применим только к отсортированным данным, поэтому может возникнуть
ситуация, когда поддержание упорядоченной структуры занимает
нерационально большое время и преимущества от бинарного поиска
нивелируются.
