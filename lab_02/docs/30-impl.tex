\chapter{Технологическая часть}

В данном разделе приведены требования к программному обеспечению, средства реализации и листинги кода.

\section{Требования к ПО}

К программе предъявляется ряд требований:
\begin{itemize}
	\item взаимодействие программы и пользователя реализованно в виде меню;
	\item пользователь может ввести матрицы для умножения ручным способом или сгенерировать, заполнив случайными числами;
	\item элементами матрицы могут быть только целые числа;
	\item считается, что пользователь вводит корректные данные;
	\item на выходе -- результат умножения матриц, вычисленный стандартным способом, алгоритмом Винограда и его оптимизированной версии.
\end{itemize}

\section{Средства реализации}

В качестве языка программирования для реализации данной лабораторной работы был выбран многопоточный язык GO \cite{golang}. Данный выбор обусловлен моим желанием расширить свои знания в области применения данного язкыа. Так же данный язык предоставляет средства тестирования разработанного ПО.

\section{Листинг кода}

В листингах \ref{lst:generator}--\ref{lst:winograd_o} приведены реализации вспомогательных утилит, а также алгоритма вычисления произведения двух матриц классическим способом, алгоритма Винограда без оптимизации и с оптимизацией.

\clearpage

\captionsetup{singlelinecheck = false, justification=raggedright}

\begin{lstinputlisting}[
	caption={Генератор матрицы},
	label={lst:generator},
	style={go},
	linerange={8-23},
	]{../src/matrix/matrix.go}
\end{lstinputlisting}

\begin{lstinputlisting}[
	caption={Функция для выделения памяти},
	label={lst:allocator},
	style={go},
	linerange={31-40},
	]{../src/matrix/matrix_io.go}
\end{lstinputlisting}

\begin{lstinputlisting}[
	caption={Классический алгоритм},
	label={lst:classic},
	style={go},
	linerange={29-39},
	]{../src/matrix/matrix.go}
\end{lstinputlisting}

\clearpage

\begin{lstinputlisting}[
	caption={Алгоритм Винограда без оптимизации},
	label={lst:winograd},
	style={go},
	linerange={41-82},
	]{../src/matrix/matrix.go}
\end{lstinputlisting}

\clearpage

\begin{lstinputlisting}[
	caption={Алгоритм Винограда c оптимизацией},
	label={lst:winograd_o},
	style={go},
	linerange={84-129},
	]{../src/matrix/matrix.go}
\end{lstinputlisting}


В таблице \ref{tabular:test_rec} приведены функциональные тесты для алгоритмов матриц. Все тесты пройдены успешно (таблица \ref{tabular:test_rec_res}).


\begin{table}[h!]
	\caption{\label{tabular:test_rec} Функциональные тесты}
	\begin{center}
		\begin{tabular}{c@{\hspace{7mm}}c@{\hspace{7mm}}c@{\hspace{7mm}}c@{\hspace{7mm}}c@{\hspace{7mm}}c@{\hspace{7mm}}}
			\hline
			Первая матрица & Вторая матрица & Ожидаемый результат \\ \hline
			\vspace{4mm}
			$\begin{pmatrix}
				1 & 2 & 3\\
				1 & 2 & 3\\
				1 & 2 & 3
			\end{pmatrix}$ &
			$\begin{pmatrix}
				1 & 2 & 3\\
				1 & 2 & 3\\
				1 & 2 & 3
			\end{pmatrix}$ &
			$\begin{pmatrix}
				6 & 12 & 18\\
				6 & 12 & 18\\
				6 & 12 & 18
			\end{pmatrix}$ \\
			\vspace{2mm}
			\vspace{2mm}
			$\begin{pmatrix}
				1 & 2 & 2\\
				1 & 2 & 2
			\end{pmatrix}$ &
			$\begin{pmatrix}
				1 & 2\\
				1 & 2\\
				1 & 2
			\end{pmatrix}$ &
			$\begin{pmatrix}
				5 & 10\\
				5 & 10
			\end{pmatrix}$ \\
			\vspace{2mm}
			\vspace{2mm}
			$\begin{pmatrix}
				2
			\end{pmatrix}$ &
			$\begin{pmatrix}
				2
			\end{pmatrix}$ &
			$\begin{pmatrix}
				4
			\end{pmatrix}$ \\
			\vspace{2mm}
			\vspace{2mm}
			$\begin{pmatrix}
				1 & -2 & 3\\
				1 & 2 & 3\\
				1 & 2 & 3
			\end{pmatrix}$ &
			$\begin{pmatrix}
				-1 & 2 & 3\\
				1 & 2 & 3\\
				1 & 2 & 3
			\end{pmatrix}$ &
			$\begin{pmatrix}
				0 & 4 & 6\\
				4 & 12 & 18\\
				4 & 12 & 18
			\end{pmatrix}$\\
			\vspace{2mm}
			\vspace{2mm}
		\end{tabular}
	\end{center}
\end{table}

\begin{table}[h!]
	\caption{\label{tabular:test_rec_res} Результат работы программы}
	\begin{center}
		\begin{tabular}{c@{\hspace{7mm}}c@{\hspace{7mm}}c@{\hspace{7mm}}c@{\hspace{7mm}}c@{\hspace{7mm}}c@{\hspace{7mm}}}
			\hline
			Первая матрица & Вторая матрица & Фактический результат \\ \hline
			\vspace{4mm}
			$\begin{pmatrix}
				1 & 2 & 3\\
				1 & 2 & 3\\
				1 & 2 & 3
			\end{pmatrix}$ &
			$\begin{pmatrix}
				1 & 2 & 3\\
				1 & 2 & 3\\
				1 & 2 & 3
			\end{pmatrix}$ &
			$\begin{pmatrix}
				6 & 12 & 18\\
				6 & 12 & 18\\
				6 & 12 & 18
			\end{pmatrix}$ \\
			\vspace{2mm}
			\vspace{2mm}
			$\begin{pmatrix}
				1 & 2 & 2\\
				1 & 2 & 2
			\end{pmatrix}$ &
			$\begin{pmatrix}
				1 & 2\\
				1 & 2\\
				1 & 2
			\end{pmatrix}$ &
			$\begin{pmatrix}
				5 & 10\\
				5 & 10
			\end{pmatrix}$ \\
			\vspace{2mm}
			\vspace{2mm}
			$\begin{pmatrix}
				2
			\end{pmatrix}$ &
			$\begin{pmatrix}
				2
			\end{pmatrix}$ &
			$\begin{pmatrix}
				4
			\end{pmatrix}$ \\
			\vspace{2mm}
			\vspace{2mm}
			$\begin{pmatrix}
				1 & -2 & 3\\
				1 & 2 & 3\\
				1 & 2 & 3
			\end{pmatrix}$ &
			$\begin{pmatrix}
				-1 & 2 & 3\\
				1 & 2 & 3\\
				1 & 2 & 3
			\end{pmatrix}$ &
			$\begin{pmatrix}
				0 & 4 & 6\\
				4 & 12 & 18\\
				4 & 12 & 18
			\end{pmatrix}$\\
			\vspace{2mm}
			\vspace{2mm}
		\end{tabular}
	\end{center}
\end{table}

\captionsetup{singlelinecheck = false, justification=centering}

\section{Вывод}

Были разработаны и протестированы спроектированные алгоритмы: вычисления произведения двух матриц классическим способом, алгоритмом Винограда без оптимизации и с оптимизацией.
