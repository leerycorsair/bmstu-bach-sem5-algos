\chapter{Исследовательская часть}

\section{Пример работы}
На рисунке \ref{img:demo} приведен пример работы программы.
\img{185mm}{demo}{Демонстрация работы программы}

\section{Технические характеристики}

Технические характеристики устройства, на котором выполнялось тестирование:

\begin{itemize}
	\item Операционная система: Windows 10 64-bit \cite{windows}.
	\item Память: 16 GB.
	\item Процессор: AMD Ryzen 5 4600H \cite{amd} @ 3.00 GHz
\end{itemize}

Тестирование проводилось на ноутбуке при включённом режиме производительности. Во время тестирования ноутбук был нагружен только системными процессами.

\section{Время выполнения алгоритмов}

Алгоритмы тестировались при помощи написания <<бенчмарков>> \cite{gotest}, предоставляемых встроенными в Go средствами. Для такого рода тестирования не нужно самостоятельно указывать количество повторов. В библиотеке для тестирования существует константа $N$, которая динамически варьируется в зависимости от того, был ли получен стабильный результат или нет.

\begin{lstinputlisting}[
	caption={Пример реализации бенчмарка},
	label={lst:matrix_test},
	style={go},
	linerange={5-16},
	]{../src/matrix/matrix_test.go}
\end{lstinputlisting}

На рисунках \ref{plt:time_even} и \ref{plt:time_odd} приведены графики зависимостей времени работы алгоритмов умножения матриц от их размеров.
\captionsetup{singlelinecheck = false, justification=raggedright}
\begin{figure}[h]
	\centering
	\begin{tikzpicture}
		\begin{axis}[
			axis lines=left,
			xlabel=Размер матрицы,
			ylabel={Время, нс},
			legend pos=north west,
			ymajorgrids=true
			]
			\addplot table[x=size,y=Classic,col sep=comma] {inc/csv/time_classic_even.csv};
			\addplot table[x=size,y=Winograd,col sep=comma] {inc/csv/time_winograd_even.csv};
			\addplot table[x=size,y=ImpWinograd,col sep=comma] {inc/csv/time_imp_winograd_even.csv};
			\legend{Классический, Виноград, Оптим. Виноград}
		\end{axis}
	\end{tikzpicture}
	\captionsetup{justification=centering}
	\caption{Зависимость времени работы алгоритма умножения матриц от размера матрицы. Четные матрицы.}
	\label{plt:time_even}
\end{figure}

\begin{figure}[h]
	\centering
	\begin{tikzpicture}
		\begin{axis}[
			axis lines=left,
			xlabel=Размер матрицы,
			ylabel={Время, нс},
			legend pos=north west,
			ymajorgrids=true
			]
			\addplot table[x=size,y=Classic,col sep=comma] {inc/csv/time_classic_odd.csv};
			\addplot table[x=size,y=Winograd,col sep=comma] {inc/csv/time_winograd_odd.csv};
			\addplot table[x=size,y=ImpWinograd,col sep=comma] {inc/csv/time_imp_winograd_odd.csv};
			\legend{Классический, Виноград, Оптим. Виноград}
		\end{axis}
	\end{tikzpicture}
	\captionsetup{justification=centering}
	\caption{Зависимость времени работы алгоритма умножения матриц от размера матрицы. Нечетные матрицы.}
	\label{plt:time_odd}
\end{figure}
\clearpage
\captionsetup{singlelinecheck = false, justification=centering}

\section{Вывод}

Исходя из полученных эксперементальных данных можно сделать несколько выводов.

На матрицах четных размеров классический алгоритм показывает худшие результаты по сравнению с алгоритмом Винограда и его оптимизированной версией (работает дольше в ~1.24 и в ~1.55 раз соответственно, т.е. около ~24\% и ~55\%). Следует отметить, что выполненные оптимизации позволили уменьшить время выполнения алгоритма Винограда в ~1.25 раз, т.е. ~25\%.

На матрицах нечетных размеров классический алгоритм также показывает показатель хуже, чем алгоритм Винограда и его оптимизированная версия. Однако отставание становится меньше, т.к. в последних приходится выполнять один дополнительный цикл. Классический алгоритм работает медленее в ~1.20 и в ~1.42 раз, т.е. около 20\% и 42\%.
