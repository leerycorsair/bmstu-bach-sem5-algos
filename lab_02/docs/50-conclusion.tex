\chapter*{Заключение}
\addcontentsline{toc}{chapter}{Заключение}

В ходе выполнения работы были выполнены все поставленные задачи и изучены методы динамического программирования на основе алгоритмов умножения матриц.

Экспериментально были подтверждены вычисленные теоритически значения трудоемкости и установлены различия в производительности различных алгоритмов.

На матрицах четных размеров классический алгоритм показывает худшие результаты по сравнению с алгоритмом Винограда и его оптимизированной версией (работает дольше в ~1.24 и в ~1.55 раз соответственно, т.е. около ~24\% и ~55\%). Следует отметить, что выполненные оптимизации позволили уменьшить время выполнения алгоритма Винограда в ~1.25 раз, т.е. ~25\%.

На матрицах нечетных размеров классический алгоритм также показывает показатель хуже, чем алгоритм Винограда и его оптимизированная версия. Однако отставание становится меньше, т.к. в последних приходится выполнять один дополнительный цикл. Классический алгоритм работает медленее в ~1.20 и в ~1.42 раз, т.е. около 20\% и 42\%.