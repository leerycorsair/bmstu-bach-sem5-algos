\chapter*{Введение}
\addcontentsline{toc}{chapter}{Введение}

Разработка и совершенствование матричных алгоритмов является важнейшей алгоритмической задачей. Непосредственное применение классического матричного умножения требует времени порядка $O(n^3)$. Однако существуют алгоритмы умножения матриц, работающие быстрее очевидного. В линейной алгебре алгоритм Копперсмита – Винограда\cite{winograd}, названный в честь Д. Копперсмита и Ш. Винограда , был асимптотически самый быстрый из известных алгоритмов умножения матриц с 1990 по 2010 год. В данной работе внимание акцентируется на алгоритме Копперсмита – Винограда и его улучшениях. 

Алгоритм не используется на практике, потому что он дает преимущество только для матриц настолько больших размеров, что они не могут быть обработаны современным вычислительным оборудованием. Если матрица не велика, эти алгоритмы не приводят к большой разнице во времени вычислений.

Целью данной лабораторной работы являются изучение и реализация алгоритмов умножения матриц с оценкой их трудоемкости.

Для достижения указанной выше цели следует выполнить следующие задачи:
\begin{itemize}
	\item изучить алгоритмы умножения матриц: стандартный алгоритм и алгоритм Винограда;
	\item оптимизировать алгоритм Винограда;
	\item дать теоритическую оценку трудоемкости классического алгоритма умножения матриц, алгоритма Винограда и улучшеного алгоритма Винограда;
	\item провести сравнительный анализ алгоритмов на основе эксперементальных данных;
	\item описание и обоснование полученных результатов в отчете о выполненной лабораторной работе.
\end{itemize}
