\chapter{Конструкторская часть}

В данном разделе представлены схемы алгоритмов, описанных в аналитическом разделе, а также описана их трудоемкость.

\section{Схемы алгоритмов}

Схема классического алгоритма приведена на рисунке \ref{img:default}.
\img{180mm}{default}{Схема классического алгоритма}

Схема алгоритма Винограда без оптимизаций приведена на рисунках \ref{img:winograd_1}-\ref{img:winograd_2}.
\img{185mm}{winograd_1}{Схема алгоритма Винограда (часть 1)}
\img{145mm}{winograd_2}{Схема алгоритма Винограда (часть 2)}
\clearpage

Схема алгоритма Винограда с оптимизациями приведена на рисунках \ref{img:winograd_optimized_1}-\ref{img:winograd_optimized_2}.
\img{185mm}{winograd_optimized_1}{Схема алгоритма Винограда с оптимизациями (часть 1)}
\img{145mm}{winograd_optimized_2}{Схема алгоритма Винограда c оптимизациями (часть 2)}
\clearpage



Для алгоритма Копперсмита -- Винограда худшим случаем являются матрицы с нечётным общим размером, а лучшим - с чётным, из-за того что отпадает необходимость в последнем цикле.

Данный алгоритм можно оптимизировать следующими методами \cite{optimisations}:
\begin{itemize}
	\item предварительно получить строки столбцы соответствующих матриц;
	\item заменить вызов функции вычисления длины вектора на заранее вычисленное значение;
	\item уменьшить количество обращений к элементам матрицы путем добавления буферной переменной;
	\item вынести конструкции if-then-else за пределы лямбда-функции.
\end{itemize}

\section{Модель вычислений}
Для последующего вычисления трудоемкости введём модель вычислений:

\begin{enumerate}
	\item Операции из списка (\ref{for:opers}) имеют трудоемкость 1.
	\begin{equation}
		\label{for:opers}
		+, -, /, \%, ==, !=, <, >, <=, >=, [], ++, {-}-
	\end{equation}
	\item Трудоемкость оператора выбора if условие then A else B рассчитывается, как (\ref{for:if}).
	\begin{equation}
		\label{for:if}
		f_{if} = f_{\text{условия}} +
		\begin{cases}
			f_A, & \text{если условие выполняется,}\\
			f_B, & \text{иначе.}
		\end{cases}
	\end{equation}
	\item Трудоемкость цикла рассчитывается, как (\ref{for:for}).
	\begin{equation}
		\label{for:for}
		f_{for} = f_{\text{инициализации}} + f_{\text{сравнения}} + N(f_{\text{тела}} + f_{\text{инкремента}} + f_{\text{сравнения}})
	\end{equation}
	\item Трудоемкость вызова функции равна 0.
\end{enumerate}

\section{Трудоемкость алгоритмов}
\subsection{Стандартный алгоритм умножения матриц}

Трудоёмкость стандартного алгоритма умножения матриц состоит из:
\begin{itemize}
	\item внешнего цикла по $i \in [1..A]$, трудоёмкость которого: $f = 2 + A \cdot (2 + f_{body})$;
	\item цикла по $j \in [1..C]$, трудоёмкость которого: $f = 2 + C \cdot (2 + f_{body})$;
	\item скалярного умножения двух векторов - цикл по $k \in [1..B]$, трудоёмкость которого: $f = 2 + 10B$.
\end{itemize}

Трудоёмкость стандартного алгоритма равна трудоёмкости внешнего цикла, можно вычислить ее, подставив циклы тела (\ref{for:base}):
\begin{equation}
	\label{for:base}
	f_{base} = 2 + A \cdot (4 + C \cdot (4 + 10B)) = 2 + 4A + 4AC + 10ABC \approx 10ABC
\end{equation}

\subsection{Алгоритм Копперсмита — Винограда}

Трудоёмкость алгоритма Копперсмита — Винограда состоит из:

\begin{enumerate}
	\item Создания векторов rows и cols (\ref{for:init}):
	\begin{equation}
		\label{for:init}
		f_{create} = A + C;
	\end{equation}
	
	\item Заполнения вектора rows (\ref{for:MH}):
	\begin{equation}
		\label{for:MH}
		f_{rows} = 3 + \frac{B}{2} \cdot (5 + 12A);
	\end{equation}
	
	\item Заполнения вектора cols (\ref{for:MV}):
	\begin{equation}
		\label{for:MV}
		f_{cols} = 3 + \frac{B}{2} \cdot (5 + 12C);
	\end{equation}
	
	\item Цикла заполнения матрицы для чётных размеров (\ref{for:cycle}):
	\begin{equation}
		\label{for:cycle}
		f_{cycle} = 2 + A \cdot (4 + C \cdot (11 + \frac{25}{2} \cdot B));
	\end{equation}
	
	\item Цикла, для дополнения умножения суммой последних нечётных строки и столбца, если общий размер нечётный (\ref{for:last}):
	\begin{equation}
		\label{for:last}
		f_{last} = \begin{cases}
			2, & \text{чётная,}\\
			4 + A \cdot (4 + 14C), & \text{иначе.}
		\end{cases}
	\end{equation}
\end{enumerate}

Итого, для худшего случая (нечётный размер матриц): 
\begin{equation}
	\label{for:bad}
	f_{wino\_w} = A + C + 12 + 8A + 5B + 6AB + 6CB + 25AC + \frac{25}{2}ABC \approx 12.5 \cdot MNK
\end{equation}

Для лучшего случая (чётный размер матриц): 
\begin{equation}
	\label{for:good}
	f_{wino\_b} = A + C + 10 + 4A + 5B + 6AB + 6CB + 11AC + \frac{25}{2}ABC \approx 12.5 \cdot MNK
\end{equation}

\subsection{Оптимизированный алгоритм Копперсмита — Винограда}

Трудоёмкость улучшенного алгоритма Копперсмита — Винограда состоит из:
\begin{enumerate}
	\item Создания векторов rows и cols (\ref{for:impr_init}):
	\begin{equation}
		\label{for:impr_init}
		f_{init} = A + C;
	\end{equation}
	
	\item Заполнения вектора rows (\ref{for:impr_MH}):
	\begin{equation}
		\label{for:impr_MH}
		f_{rows} = 2 + \frac{B}{2} \cdot (4 + 8A);
	\end{equation}
	
	\item Заполнения вектора cols (\ref{for:impr_MV}):
	\begin{equation}
		\label{for:impr_MV}
		f_{cols} = 2 + \frac{B}{2} \cdot (4 + 8A);
	\end{equation}
	
	\item Цикла заполнения матрицы для чётных размеров (\ref{for:impr_cycle}):
	\begin{equation}
		\label{for:impr_cycle}
		f_{cycle} = 2 + A \cdot (4 + C \cdot (8 + 9B))
	\end{equation}
	
	\item Цикла, для дополнения умножения суммой последних нечётных строки и столбца, если общий размер нечётный (\ref{for:impr_last}):
	\begin{equation}
		\label{for:impr_last}
		f_{last} = 
		\begin{cases}
			2, & \text{чётная,}\\
			4 + A \cdot (4 + 12C), & \text{иначе.}
		\end{cases}
	\end{equation}
\end{enumerate}

Итого, для худшего случая (нечётный общий размер матриц) имеем (\ref{for:bad_impr}):
\begin{equation}
	\label{for:bad_impr}
	f = A + C + 10 + 4B + 4BC + 4BA + 8A + 20AC + 9ABC \approx 9ABC
\end{equation}

Для лучшего случая (чётный общий размер матриц) имеем (\ref{for:good_impr}):
\begin{equation}
	\label{for:good_impr}
	f = A + C + 8 + 4B +4BC + 4BA + 4A + 8AC + 9ABC \approx 9ABC
\end{equation}
\section{Вывод}

На основе теоретических данных, полученных из аналитического раздела, были построены схемы требуемых алгоритмов и проведена теоретическая оценка их трудоемкости.
