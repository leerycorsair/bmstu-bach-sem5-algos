\chapter{Аналитическая часть}

\section{Конвейерная обработка данных}
Пусть операция разбита на микрооперации. Расположим микрооперации в порядке выполнения и для каждого выполнения выделим отдельную часть устройства. В первый момент времени входные данные поступают для обработки в первую часть. После выполнения первой микрооперации первая часть передает результаты своей работы второй части, а сама берет новые данные. Когда входные аргументы пройдут все этапы обработки, на выходе устройства появится результат выполнения операции. Таким образом, реализуется функциональный параллелизм. Каждая часть устройства называется ступенью конвейера, а общее число ступеней – длиной конвейера.

Каждая из ступеней конвейера использует отдельный поток и очередь для организации корректной работы системы.

\section{Описание предметной области}
В качестве алгоритма, реализованного для распределения на конвейере, был выбран абстрактный процесс генерации дневного рациона пищи для клиентов службы доставки еды:
\begin{itemize}
	\item процесс подбора блюда на завтрак;
	\item процесс подбора блюда на обед;
	\item процесс подбора блюда на ужин.
\end{itemize}

\section{Вывод}

Для ПО, решающего поставленную задачу следует выделить следующие требования:
\begin{itemize}
	\item ПО должно генерировать корректное меню дневного рациона;
	\item ПО должно генерировать один дневной рацион не более чем за 50 мс;
	\item ПО должно обеспечить методы контроля времени обработки каждого из этапов конвейера.
\end{itemize}

Были рассмотрены принципы и особенности построения конвейерных вычислений, а также подобрана задача для моделирования поставленной задачи. 


