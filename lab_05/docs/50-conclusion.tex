\chapter*{Заключение}
\addcontentsline{toc}{chapter}{Заключение}

В ходе выполнения работы были выполнены все поставленные задачи и изучены методы динамического многопоточного программирования на основе алгоритма генерации дневного рациона для клентов службы доставки еды.

Реализация многопоточного алгоритма на основе конвейейрного подхода позволяет получить выигрыш по быстродействию порядка ~2.87 раз (при условии, что рассматриваемый алгоритм состоит из трех независимых алгоритмов, каждый из которых обрабатывается на отдельном потоке).

При реализации многопоточного алгоритма наблюдается многократное уменьшение времени простоя каждой из лент конвейера по сравнению с его линейным вариантом (приблизительно в 1000 раз).

Параллельные конвейерные вычисления позволяют организовать непрерывную обработку данных, что позволяет выиграть время в задачах, где требуется обработка больших объемов информации за небольшой промежуток времени.

