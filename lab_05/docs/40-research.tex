\chapter{Исследовательская часть}

\section{Пример работы}

Демонстрация работы программы приведена на рисунке \ref{img:demo}.

\boximg{140mm}{demo}{Демонстрация работы алгоритмов сортировок}

\section{Технические характеристики}

Технические характеристики устройства, на котором выполнялось тестирование:

\begin{itemize}
	\item операционная система: Windows 10 64-bit \cite{windows};
	\item память: 16 GB;
	\item процессор: AMD Ryzen 5 4600H \cite{amd} @ 3.00 GHz.
\end{itemize}

Тестирование проводилось на ноутбуке при включённом режиме производительности. Во время тестирования ноутбук был нагружен только системными процессами.

\section{Время выполнения алгоритмов}

В таблицах \ref{tbl:l_time}-\ref{tbl:p_time} приведены временные отметки работы линейного и параллельного конвейера.

\begin{table}[h]
	\begin{center}
		\caption{Лог обработки заявок на линейном конвейере}
		\label{tbl:l_time}
		\begin{tabular}{|c|c|c|c|}
			\hline
			№ заявки&№ этапа&Время начала&Время конца \\ \hline
 1&1&0s&11.310012ms\\  1&2&11.311935ms&13.47765ms\\  1&3&13.478612ms&22.957808ms\\ \hline 2&1&22.960773ms&28.122197ms\\  2&2&28.123219ms&34.346058ms\\  2&3&34.34691ms&34.349064ms\\ \hline 3&1&34.350367ms&50.734355ms\\  3&2&50.735587ms&61.251483ms\\  3&3&61.252795ms&66.435439ms\\ \hline 4&1&66.438625ms&81.97257ms\\  4&2&81.974224ms&85.301686ms\\  4&3&85.303239ms&88.611855ms\\ \hline 5&1&88.633807ms&96.07894ms\\  5&2&96.080503ms&104.570381ms\\  5&3&104.571714ms&114.104312ms\\ \hline
		\end{tabular}
	\end{center}
\end{table}
\clearpage

\begin{table}[h]
	\begin{center}
		\caption{Лог обработки заявок на параллельном конвейере}
		\label{tbl:p_time}
		\begin{tabular}{|c|c|c|c|}
			\hline
			№ заявки&№ этапа&Время начала&Время конца \\ \hline
			 1&1&0s&19.372072ms\\  1&2&19.447747ms&34.107102ms\\  1&3&34.169231ms&50.547227ms\\ \hline 2&1&19.393663ms&21.755132ms\\  2&2&34.165313ms&50.508273ms\\  2&3&50.54841ms&56.741513ms\\ \hline 3&1&21.756845ms&23.112743ms\\  3&2&50.510327ms&67.960691ms\\  3&3&68.045643ms&75.5349ms\\ \hline 4&1&23.114296ms&42.775641ms\\  4&2&67.964217ms&71.204754ms\\  4&3&75.537375ms&79.846261ms\\ \hline 5&1&42.779418ms&45.396334ms\\  5&2&71.206617ms&72.406458ms\\  5&3&79.848515ms&97.096062ms\\ \hline
		\end{tabular}
	\end{center}
\end{table}



На рисунке \ref{plt:time} приведен график зависимости времени работы алгоритмов для различных наборов данных.

\begin{figure}[!h]
	\centering
	\begin{tikzpicture}
		\begin{axis}[
			axis lines=left,
			xlabel=Количество заявок,
			ylabel={Время, мс},
			legend pos=north west,
			ymajorgrids=true
			]
			\addplot table[x=recs,y=time,col sep=comma] {inc/csv/linear.csv};
			\addplot table[x=recs,y=time,col sep=comma] {inc/csv/parallel.csv};
			\legend{Линейный, Параллельный}
		\end{axis}
	\end{tikzpicture}
	\captionsetup{justification=centering}
	\caption{Сравнение скорости работы алгоритмов конвейера при различном количестве заявок.}
	\label{plt:time}
\end{figure}


\captionsetup{singlelinecheck = false, justification=centering}


\section{Вывод}

Исходя из полученных эксперементальных данных можно сделать следующие выводы:
\begin{itemize}
	\item реализация многопоточного алгоритма на основе конвейейрного подхода позволяет получить выигрыш по быстродействию порядка ~2.87 раз (при условии, что рассматриваемый алгоритм состоит из трех независимых алгоритмов, каждый из которых обрабатывается на отдельном потоке);
	\item при реализации многопоточного алгоритма наблюдается многократное уменьшение времени простоя каждой из лент конвейера по сравнению с его линейным вариантом (приблизительно в 1000 раз).
\end{itemize}


