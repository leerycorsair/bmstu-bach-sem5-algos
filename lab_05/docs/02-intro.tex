\chapter*{Введение}
\addcontentsline{toc}{chapter}{Введение}

21 век можно по праву назвать веком многоядерных машин. Многоядерность позволяет существенным образом повысить быстродействие систем. Существует множество подходов для использования многоядерности для ускорения вычислений. В ситуациях, когда один и тот же набор данных необходимо обработать несколькими независимыми алгоритмами, используют конвейерный подход по аналогии с традиционным подходом производства на фабриках. На каждый этап обработки данных выделяется отдельная конвейерная лента, которая отвечает за конкретную подзадачу \cite{parallel}.

Целью данной лабораторной работы являются изучение и реализация конвейерных вычислений.

Для достижения указанной выше цели следует выполнить следующие задачи:
\begin{itemize}
	\item рассмотреть и изучить асинхронную конвейерную обработку данных;
	\item сформулировать задачу, которая будет реализованна на конвейере;
	\item определить требования к ПО;
	\item построить схемы алгоритмов, решающих поставленную задачу с использованием последовательного и параллельного конвейера;
	\item описать структуру ПО;
	\item исходя из полученных эксперементально данных, провести сравнительный анализ алгоритмов, реализованных линейно и параллельно;
	\item описание и обоснование полученных результатов в отчете о выполненной лабораторной работе.
\end{itemize}

