\chapter{Технологическая часть}

В данном разделе приведены требования к программному обеспечению, средства реализации и листинги кода.

\section{Требования к ПО}

К программе предъявляется ряд требований:
\begin{itemize}
	\item на вход подается количество заявок на создание дневного рациона;
	\item на выходе лог обработки заявок с временными метками;
	\item существует система автоматического тестирования.
\end{itemize}

\section{Средства реализации}

В качестве языка программирования для реализации данной лабораторной работы был выбран многопоточный язык GO \cite{golang}. Данный выбор обусловлен моим желанием расширить свои знания в области применения данного языка. Также данный язык предоставляет удобные средства для работы с потоками (goroutines) \cite{goconc}.
 
Время работы алгоритмов было замерено с помощью функции {\ttfamily Now()}
из библиотеки {\ttfamily Time}.

\section{Листинг кода}

В листингах \ref{lst:linear}--\ref{lst:gen} приведены  линейная и конвейерная реализации алгоритма, а также вспомогательные функции.

\clearpage

\captionsetup{singlelinecheck = false, justification=raggedright}

\begin{lstinputlisting}[
	caption={Линейный алгоритм},
	label={lst:linear},
	style={go},
	linerange={5-42},
	]{../src/conveyor/linear.go}
\end{lstinputlisting}

\clearpage

\begin{lstinputlisting}[
	caption={Конвейерный алгоритм},
	label={lst:parallel},
	style={go},
	linerange={8-54},
	]{../src/conveyor/parallel.go}
\end{lstinputlisting}

\clearpage

\begin{lstinputlisting}[
	caption={Алгоритмы логирования},
	label={lst:log},
	style={go},
	linerange={8-46},
	]{../src/conveyor/log.go}
\end{lstinputlisting}

\clearpage

\begin{lstinputlisting}[
	caption={Используемые структуры данных},
	label={lst:structs},
	style={go},
	linerange={5-31},
	]{../src/conveyor/types.go}
\end{lstinputlisting}

\begin{lstinputlisting}[
	caption={Алгоритмы обработки очереди},
	label={lst:queue},
	style={go},
	linerange={5-17},
	]{../src/conveyor/queue.go}
\end{lstinputlisting}
\clearpage



\captionsetup{singlelinecheck = false, justification=centering}

\section{Вывод}

Были разработаны линейный и конвейерный спроектированные алгоритмы генерации дневного рациона клиента службы доставки.
