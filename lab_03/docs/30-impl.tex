\chapter{Технологическая часть}

В данном разделе приведены требования к программному обеспечению, средства реализации и листинги кода.

\section{Требования к ПО}

К программе предъявляется ряд требований:
\begin{itemize}
	\item взаимодействие программы и пользователя реализованно в виде меню;
	\item пользователь может ввести массив для сортировки ручным способом или сгенерировать, заполнив случайными значениями;
	\item элементами массива могут быть только целые числа;
	\item считается, что пользователь вводит корректные данные;
	\item на выходе — отсортированный массив каждым из методов.
\end{itemize}

\section{Средства реализации}

В качестве языка программирования для реализации данной лабораторной работы был выбран многопоточный язык GO \cite{golang}. Данный выбор обусловлен моим желанием расширить свои знания в области применения данного язкыа. Так же данный язык предоставляет средства тестирования разработанного ПО.

\section{Листинг кода}

В листингах \ref{lst:bubble}--\ref{lst:insertion} приведены реализации алгоритмов сортировок, а также вспомогательные функции.

\clearpage

\captionsetup{singlelinecheck = false, justification=raggedright}

\begin{lstinputlisting}[
	caption={Алгоритм сортировки пузырьком},
	label={lst:bubble},
	style={go},
	linerange={25-38},
	]{../src/array/array.go}
\end{lstinputlisting}

\begin{lstinputlisting}[
	caption={Алгоритм сортировки выбором},
	label={lst:selection},
	style={go},
	linerange={51-63},
	]{../src/array/array.go}
\end{lstinputlisting}

\begin{lstinputlisting}[
	caption={Алгоритм сортировки вставками},
	label={lst:insertion},
	style={go},
	linerange={40-49},
	]{../src/array/array.go}
\end{lstinputlisting}

\clearpage



В таблице \ref{tabular:func_test} приведены функциональные тесты для алгоритмов сортировки. Все тесты пройдены успешно (таблица \ref{tabular:func_test_res}).


\begin{table}[h!]
	\begin{center}
		\caption{\label{tabular:func_test}Ожидаемый результат работы программы}
		\begin{tabular}{|c|c|}
			\hline
			Входной массив & Ожидаемый результат \\ 
			\hline
			$[31, 68, 115, 124, 155, 173]$ & $[31, 68, 115, 124, 155, 173]$ \\\hline
			$[31, -6, -53, -62, -93, -111, -136]$ & $[31, -6, -53, -62, -93, -111, -136]$ \\\hline
			$[31, 37, 47, 9, 31, 18, 25]$ & $[9, 18, 25, 31, 31, 37, 47]$ \\\hline
		\end{tabular}
	\end{center}
\end{table}

\begin{table}[h!]
	\begin{center}
		\caption{\label{tabular:func_test_res}Фактический результат работы программы}
		\begin{tabular}{|c|c|}
			\hline
			Входной массив & Фактический результат \\ 
			\hline
			$[31, 68, 115, 124, 155, 173]$ & $[31, 68, 115, 124, 155, 173]$ \\\hline
			$[31, -6, -53, -62, -93, -111, -136]$ & $[31, -6, -53, -62, -93, -111, -136]$ \\\hline
			$[31, 37, 47, 9, 31, 18, 25]$ & $[9, 18, 25, 31, 31, 37, 47]$ \\\hline
		\end{tabular}
	\end{center}
\end{table}


\captionsetup{singlelinecheck = false, justification=centering}

\section{Вывод}

Были разработаны и протестированы спроектированные алгоритмы: алгоритм сортировки пузырьком, алгоритм сортировки вставками и алгоритм сортировки выбором.
