\chapter*{Введение}
\addcontentsline{toc}{chapter}{Введение}

Сортировка является одной из важнейших операций в области обработки данных\cite{knut}. Сортировкой называют процесс перегруппировки заданной последовательности объектов в некотором определенном порядке. Упорядоченность в последовательности объектов необходимо для удобства работы с этим объектом и значительного упрощения некоторых алгоритмов засчёт полученной упорядоченности (например, поиск элементов с определёнными значениями каких-то характеристик в массиве).

Алгоритмы сортировки используются практически в любой программной системе. Целью алгоритмов сортировки является упорядочение последовательности элементов данных. Поиск элемента в последовательности отсортированных данных занимает время, пропорциональное логарифму количеству элементов в последовательности, а поиск элемента в последовательности неотсортированных данных занимает время, пропорциональное количеству элементов в последовательности, что существенно больше. Существует множество различных методов сортировки данных. Однако любой алгоритм сортировки можно разбить на три основные части:

\begin{itemize}
	\item сравнение, определяющее упорядоченность пары элементов;
	\item перестановка, меняющая местами пару элементов;
	\item сортирующий алгоритм, который осуществляет сравнение и перестановку элементов данных до тех пор, пока все эти элементы не будут упорядочены.
\end{itemize}

Важнейшей характеристикой любого алгоритма сортировки является скорость его работы, которая определяется функциональной зависимостью среднего времени сортировки последовательностей элементов данных, заданной длины, от этой длины. Время сортировки будет пропорционально количеству сравнений и перестановки элементов данных в процессе их сортировки. Как уже было сказано, в любой сфере, использующей какое-либо программное обеспечение, с большой долей вероятности используются сортировки.  

Целью данной лабораторной работы являются изучение и реализация алгоритмов сортировки c оценкой их трудоемкости.

Для достижения указанной выше цели следует выполнить следующие задачи:
\begin{itemize}
	\item изучить и реализовать 3 алгоритма сортировки: пузырьком, вставками, выбором;
	\item привести схемы указанных алгоритмов сортировки;
	\item провести сравнительный анализ трудоёмкости алгоритмов на основе теоретических расчётов и выбранной модели вычислений;
	\item провести сравнительный анализ алгоритмов на основе экспериментальных данных;
	\item описание и обоснование полученных результатов в отчете о выполненной лабораторной работе.
\end{itemize}

