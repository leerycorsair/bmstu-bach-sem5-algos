\chapter{Исследовательская часть}

\section{Пример работы}

Демонстрация работы программы приведена на рисунке \ref{img:demo}.

\boximg{80mm}{demo}{Демонстрация работы алгоритмов сортировок}

\section{Технические характеристики}

Технические характеристики устройства, на котором выполнялось тестирование:

\begin{itemize}
	\item Операционная система: Windows 10 64-bit \cite{windows}.
	\item Память: 16 GB.
	\item Процессор: AMD Ryzen 5 4600H \cite{amd} @ 3.00 GHz.
\end{itemize}

Тестирование проводилось на ноутбуке при включённом режиме производительности. Во время тестирования ноутбук был нагружен только системными процессами.

\section{Время выполнения алгоритмов}

Алгоритмы тестировались при помощи написания <<бенчмарков>> \cite{gotest}, предоставляемых встроенными в Go средствами. Для такого  тестирования не нужно самостоятельно указывать количество повторов. В библиотеке для тестирования существует константа $N$, которая динамически для получения стабильного результата. На рисунках \ref{plt:time_rand} - \ref{plt:time_desc} приведены графики зависимостей времени работы алгоритмов для различных наборов данных.

\begin{figure}[!h]
	\centering
	\begin{tikzpicture}
		\begin{axis}[
			axis lines=left,
			xlabel=Размер массива,
			ylabel={Время, нс},
			legend pos=north west,
			ymajorgrids=true
			]
			\addplot table[x=size,y=BubbleSort,col sep=comma] {inc/csv/bubble_rand.csv};
			\addplot table[x=size,y=InsertSort,col sep=comma] {inc/csv/insertion_rand.csv};
			\addplot table[x=size,y=SelectionSort,col sep=comma] {inc/csv/selection_rand.csv};
			\legend{Bubble, Insert, Selection}
		\end{axis}
	\end{tikzpicture}
	\captionsetup{justification=centering}
	\caption{Сравнение алгоритмов на массивах случайных данных.}
	\label{plt:time_rand}
\end{figure}

\begin{figure}[!h]
	\centering
	\begin{tikzpicture}
		\begin{axis}[
			axis lines=left,
			xlabel=Размер массива,
			ylabel={Время, нс},
			legend pos=north west,
			ymajorgrids=true
			]
			\addplot table[x=size,y=BubbleSort,col sep=comma] {inc/csv/bubble_asc.csv};
			\addplot table[x=size,y=InsertSort,col sep=comma] {inc/csv/insertion_asc.csv};
			\addplot table[x=size,y=SelectionSort,col sep=comma] {inc/csv/selection_asc.csv};
			\legend{Bubble, Insert, Selection}
		\end{axis}
	\end{tikzpicture}
	\captionsetup{justification=centering}
	\caption{Сравнение алгоритмов на массивах данных, упорядоченных в прямом порядке (лучший случай).}
	\label{plt:time_asc}
\end{figure}

\begin{figure}[!h]
	\centering
	\begin{tikzpicture}
		\begin{axis}[
			axis lines=left,
			xlabel=Размер массива,
			ylabel={Время, нс},
			legend pos=north west,
			ymajorgrids=true
			]
			\addplot table[x=size,y=BubbleSort,col sep=comma] {inc/csv/bubble_desc.csv};
			\addplot table[x=size,y=InsertSort,col sep=comma] {inc/csv/insertion_desc.csv};
			\addplot table[x=size,y=SelectionSort,col sep=comma] {inc/csv/selection_desc.csv};
			\legend{Bubble, Insert, Selection}
		\end{axis}
	\end{tikzpicture}
	\captionsetup{justification=centering}
	\caption{Сравнение алгоритмов на массивах данных, упорядоченных в обратном порядке (худший случай).}
	\label{plt:time_desc}
\end{figure}


\captionsetup{singlelinecheck = false, justification=centering}


\section{Вывод}

Исходя из полученных эксперементальных данных можно сделать следующие выводы:
\begin{itemize}
	\item На массивах случайных данных алгоритм сортировки пузырьком показывает результат хуже, чем алгоритмы вставки и выбора (проигрыш порядка ~84\%  и ~43\% соответственно). Алгоритм вставки в свою очередь более эффективен, чем алгоритм выбора (выигрышь порядка ~72\%).
	\item На массивах данных, упорядоченных в прямом порядке (лучший случай) алгоритмы сортировки пузырьком и вставками показывают идентичный результат, а алгоритм выбора работает медленее, чем они в ~1560 раз. 
	\item На массивах данных, упорядоченных в обратном порядке (худший случай алгоритмы сортировки вставками и выбором показывают сопоставимые результаты (последнее в среднее медленее на ~5-10\%), а алгоритм сортировки пузырьком имеет проигрышь ~65\% и ~60\% соответсвенно. 

\end{itemize}


