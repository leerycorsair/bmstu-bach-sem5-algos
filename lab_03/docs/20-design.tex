\chapter{Конструкторская часть}

В данном разделе представлены схемы алгоритмов, описанных в аналитическом разделе, а также описана их трудоемкость.

\section{Схемы алгоритмов}

На рисунке \ref{img:bubble} приведена схема алгоритма сортировки пузырьком.

\img{150mm}{bubble}{Схема алгоритма сортировки пузырьком}

\clearpage

На рисунке \ref{img:selection} приведена схема алгоритма сортировки выбором.

\img{170mm}{selection}{Схема алгоритма сортировки выбором}

\clearpage

На рисунке \ref{img:insertion} приведена схема алгоритма сортировки вставками.

\img{170mm}{insertion}{Схема алгоритма сортировки вставками}

\section{Модель вычислений}

За одну операцию считается обмен элементов в массиве и сравнение двух элементов.

\section{Трудоёмкость алгоритмов}

Пусть размер массивов во всех вычислениях обозначается как $N$.

\subsection{Алгоритм сортировки пузырьком}

Рассмотрим худший случай. На первой итерации алгоритм делает $N-1$ сравнений, на второй $N-2$, ..., на $N-2$ итерации $1$ сравнение. Тогда общее количество сравнений:
\begin{equation}
	comp = (N-1) + (N-2) + ... + 2 + 1 = \frac{N \cdot (N-1)}{2}
\end{equation}
Каждому сравнению в худшем случае будет соответствовать такое же количество обменов:
\begin{equation}
	exch = \frac{N \cdot (N-1)}{2}
\end{equation}
Тогда общее количество операций:
\begin{equation}
	oper = 2 \cdot \frac{N \cdot (N-1)}{2} = N \cdot (N-1)
\end{equation}
В итоге, сложность алгоритма определяется как:
\begin{equation}
	O(N^2 - N) = O(N^2)
\end{equation}

Рассмотрим лучший случай. На первой итерации алгоритм делает $N-1$ сравнений, после чего обнаруживается, что обменов произведено не было и алгоритм завершается. Тогда общее количество сравнений и обменов:
\begin{align}
	comp &= (N-1) \\
	exch &= 0
\end{align}
Тогда общее количество операций:
\begin{equation}
	oper = (N-1) + 0 = (N-1)
\end{equation}
В итоге, сложность алгоритма определяется как:
\begin{equation}
	O(N-1) = O(N)
\end{equation}

Рассмотрим средний случай. Первая половина исходных данных отсортирована, другая нет. Тогда алгоритм произведёт $\frac{N-1}{2}$ сравнений, после чего обнаружит, что на предыдущей итерации обменов не было произведено, и завершится. Тогда общее количество сравнений:
\begin{equation}
	comp = (\frac{N}{2}-1) + (\frac{N}{2}-2) + ... + 2 + 1 = \frac{\frac{N}{2} \cdot (\frac{N}{2}-1)}{2}
\end{equation}
Каждому сравнению будет соответствовать такое же количество обменов:
\begin{equation}
	exch = \frac{\frac{N}{2} \cdot (\frac{N}{2}-1)}{2}
\end{equation}
Тогда общее количество операций:
\begin{equation}
	oper = 2 \cdot \frac{\frac{N}{2} \cdot (\frac{N}{2}-1)}{2} = \frac{N}{2} \cdot (\frac{N}{2}-1) = \frac{N^2}{4} - \frac{N}{2}
\end{equation}
В итоге, сложность алгоритма определяется как:
\begin{equation}
	O(\frac{N^2}{4} - \frac{N}{2}) = O(N^2)
\end{equation}

\subsection{Алгоритм сортировки вставками}

Рассмотрим худший случай. На первой итерации алгоритм делает $1$ сравнение, на второй $2$ сравнения, ..., на $N-1$-ой итерации $N-1$ сравнение. Тогда общее количество сравнений:
\begin{equation}
	comp = 1 + 2 + ... + (N-2) + (N-1) = \frac{N \cdot (N-1)}{2}
\end{equation}
Каждому сравнению в худшем случае будет соответствовать такое же количество обменов:
\begin{equation}
	exch = \frac{N \cdot (N-1)}{2}
\end{equation}
Тогда общее количество операций:
\begin{equation}
	oper = 2 \cdot \frac{N \cdot (N-1)}{2} = N \cdot (N-1)
\end{equation}
В итоге, сложность алгоритма определяется как:
\begin{equation}
	O(N^2 - N) = O(N^2)
\end{equation}

Рассмотрим лучший случай. На первой итерации алгоритм делает $1$ сравнение, на второй $1$ сравнение и выходит (так как два любые последовательных элемента отсортированы и условие входа в ветвление не выполняется), ..., на $N-1$-ой итерации $1$ сравнение и выходит. Тогда общее количество сравнений:
\begin{equation}
	comp = N \cdot 1 = N
\end{equation}
Так как обменов произведено не было, то:
\begin{equation}
	exch = 0
\end{equation}
Тогда общее количество операций:
\begin{equation}
	oper = N + 0 = N
\end{equation}
В итоге, сложность алгоритма определяется как:
\begin{equation}
	O(N)
\end{equation}

Рассмотрим средний случай. Первая половина исходных данных отсортирована, другая -- обратно отсортирована. Тогда алгоритм произведёт по $1$ сравнению до того, как дойдёт до середины массива (так как не будет выполняться условие ветвления из-за сортированности любых двух элементов первой половины), не производя обменов. После этого массив будет делать от $1$ сравнения и $0$ обменов для элемента, следующего за серединным, $2$ сравнений и $1$ обмена для следующего, и так далее до $N-1$ сравнений и $N-2$ обменов для последнего.
Тогда общее количество операций для сортированной части:
\begin{equation}
	oper_{sorted} = N \cdot 1 + 0
\end{equation}
Для обратно сортированной:
\begin{multline}
	oper_{reverse} = (1 + 2 + ... + N-1) + (0 + 1 + ... + N-2)  = \\
	= \frac{(N-1) \cdot (2N-2)}{2} = \frac{2 \cdot (N-1)^2}{2} = (N-1)^2
\end{multline}
Тогда общее количество операций:
\begin{equation}
	oper = (N-1)^2 + N
\end{equation}
В итоге, сложность алгоритма определяется как:
\begin{equation}
	O((N-1)^2 + N) = O(N^2)
\end{equation}


\subsection{Алгоритм сортировки выбором}

Рассмотрим худший случай. На первой итерации алгоритм делает $N-1$ сравнение, на второй $N-2$ сравнения, ..., на $N-2$-ой итерации $1$ сравнение. Тогда общее количество сравнений:
\begin{equation}
	comp = (N-1) + (N-2) + ... + 2 + 1 = \frac{N \cdot (N-1)}{2}
\end{equation}
По завершению сравнений на каждой итерации происходит один обмен, так как итераций $N$, то количество обменов:
\begin{equation}
	exch = N
\end{equation}
Тогда общее количество операций:
\begin{equation}
	oper = N + \frac{N \cdot (N-1)}{2}
\end{equation}
В итоге, сложность алгоритма определяется как:
\begin{equation}
	O(N + \frac{N \cdot (N-1)}{2}) = O(N^2)
\end{equation}

Рассмотрим лучший случай. На первой итерации алгоритм делает $N-1$ сравнение, на второй $N-2$ сравнения, ..., на $N-2$-ой итерации $1$ сравнение. Тогда общее количество сравнений:
\begin{equation}
	comp = (N-1) + (N-2) + ... + 2 + 1 = \frac{N \cdot (N-1)}{2}
\end{equation}
По завершению сравнений обменов не происходит, тогда:
\begin{equation}
	exch = 0
\end{equation}
Тогда общее количество операций:
\begin{equation}
	oper = 0 + \frac{N \cdot (N-1)}{2}
\end{equation}
В итоге, сложность алгоритма определяется как:
\begin{equation}
	O(0 + \frac{N \cdot (N-1)}{2}) = O(N^2)
\end{equation}

Рассмотрим средний случай. На первой итерации алгоритм делает $N-1$ сравнение, на второй $N-2$ сравнения, ..., на $N-2$-ой итерации $1$ сравнение. Тогда общее количество сравнений:
\begin{equation}
	comp = (N-1) + (N-2) + ... + 2 + 1 = \frac{N \cdot (N-1)}{2}
\end{equation}
В половине случаев необходим обмен, тогда:
\begin{equation}
	exch = \frac{N}{2}
\end{equation}
Тогда общее количество операций:
\begin{equation}
	oper = \frac{N}{2} + \frac{N \cdot (N-1)}{2} = \frac{N^2}{2}
\end{equation}
В итоге, сложность алгоритма определяется как:
\begin{equation}
	O(\frac{N^2}{2}) = O(N^2)
\end{equation}


\section{Вывод}

На основе теоретических данных, полученных из аналитического раздела, были построены схемы требуемых алгоритмов.

Лучшая асимпотическая сложность наблюдается:
\begin{itemize}
	\item В лучшем случае: у сортировок пузырьком и вставками -- $O(N)$ сравнений и $O(1)$ обменов; у сортировки выбором -- $O(N^2)$ сравнений и $O(N)$ обменов;
	\item В худшем случае: у сортировки выбором -- $O(N^2)$ сравнений и $O(N)$ обменов; у сортировок пузырьком и вставками -- $O(N^2)$ сравнений и $O(N^2)$ обменов;
	\item В среднем случае: у сортировки выбором -- $O(N^2)$ сравнений и $O(N)$ обменов; у сортировок пузырьком и вставками -- $O(N^2)$ сравнений и $O(N^2)$ обменов;
\end{itemize}

В ячейках таблицы указаны через дробь: количество сравнений / количество обменов
\begin{center}
	\begin{tabular}{| l | c | c | c |}
		\hline
		Случай & Пузырёк & Вставки & Выбор \\ \hline
		Лучший & $O(N)$/$O(1)$ & $O(N)$/$O(1)$ & $O(N^2)$/$O(N)$ \\ \hline
		Худший & $O(N^2)$/$O(N^2)$ & $O(N^2)$/$O(N^2)$ & $O(N^2)$/$O(N)$ \\ \hline
		Средний & $O(N^2)$/$O(N^2)$ & $O(N^2)$/$O(N^2)$ & $O(N^2)$/$O(N)$ \\
		\hline
	\end{tabular}
\end{center}
