\chapter{Аналитическая часть}

\section{Сортировка пузырьком}

Алгоритм состоит из повторяющихся проходов по сортируемому массиву.
За каждый проход элементы последовательно сравниваются попарно и, если порядок в паре неверный, выполняется обмен местами данной пары элементов.
Проходы по массиву повторяются $N-1$ раз, но алгоритм можно оптимизировать, проверяя на каждом проходе необходимость перестановок. Если на каком-то проходе окажется, что перестановок за весь проход не было -- массив отсортирован, работа алгоритма прекращается.
При каждом проходе алгоритма по внутреннему циклу очередной наибольший элемент массива ставится на свое место в конце массива рядом с предыдущим ``наибольшим элементом'', а наименьший элемент массива перемещается на одну позицию к началу массива (``всплывает'' до нужной позиции, как пузырёк в воде -- отсюда и название алгоритма).

\section{Сортировка вставками}

Сортировка вставками -- алгоритм сортировки, в котором элементы входной последовательности рассматриваются по одному, и каждый новый поступивший элемент размещается в подходящее место среди ранее упорядоченных элементов.

В начальный момент отсортированная последовательность пуста.
На каждом шаге алгоритма выбирается один из элементов входных данных и помещается на нужную позицию в уже отсортированной последовательности до тех пор, пока набор входных данных не будет исчерпан.
В любой момент времени в отсортированной последовательности элементы удовлетворяют требованиям к выходным данным алгоритма.

Для отбрасывания необходимости в дополнительной памяти сортировка проходит ``на месте'', отсортированной последовательностью является подпоследовательность, которую алгоритм уже обработал: на начальном шаге это $[0, 0]$, на втором (после обработки первых двух элементов) -- $[0, 1]$, и т.д.

\section{Сортировка выбором}

Сортировка выбором очень схожа с ранее рассмотренной сортировкой вставками. Однако если в сортировке вставками каждый элемент из неотсортированной подпоследовательности ставится в ``своё'' упорядоченное место в отсортированной, то в сортировке выбором отсортированная подпоследовательность строится выбором на каждом шаге минимального элемента из неотсортированной и перемещением его в конец отсортированной подпоследовательности.


\section{Вывод}

В данном разделе были рассмотрены три алгоритма сортировки массива: алгоритм сортировки пузырьком, алгоритм сортировки вставками и алгоритм сортировки выбором. Для каждой из сортировок были выделены принципы работы с последовательностями.


