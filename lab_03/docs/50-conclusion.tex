\chapter*{Заключение}
\addcontentsline{toc}{chapter}{Заключение}

В ходе выполнения работы были выполнены все поставленные задачи и изучены методы динамического программирования на основе алгоритмов сортировки массивов.

Теоритически был проведен анализ трудоемкости алгоритмов. В лучшем случае: у сортировок пузырьком и вставками -- $O(N)$ сравнений и $O(1)$ обменов; у сортировки выбором -- $O(N^2)$ сравнений и $O(N)$ обменов. В худшем случае: у сортировки выбором -- $O(N^2)$ сравнений и $O(N)$ обменов; у сортировок пузырьком и вставками -- $O(N^2)$ сравнений и $O(N^2)$ обменов. В среднем случае: у сортировки выбором -- $O(N^2)$ сравнений и $O(N)$ обменов; у сортировок пузырьком и вставками -- $O(N^2)$ сравнений и $O(N^2)$ обменов.

Также экспериментально были установлены различия алгоритмов в производительности. На массивах случайных данных алгоритм сортировки пузырьком показывает результат хуже, чем алгоритмы вставки и выбора (проигрыш порядка ~84\%  и ~43\% соответственно). Алгоритм вставки в свою очередь более эффективен, чем алгоритм выбора (выигрышь порядка ~72\%). На массивах данных, упорядоченных в прямом порядке (лучший случай) алгоритмы сортировки пузырьком и вставками показывают идентичный результат, а алгоритм выбора работает медленее, чем они в ~1560 раз. На массивах данных, упорядоченных в обратном порядке (худший случай алгоритмы сортировки вставками и выбором показывают сопоставимые результаты (последнее в среднее медленее на ~5-10\%), а алгоритм сортировки пузырьком имеет проигрышь ~65\% и ~60\% соответсвенно.

