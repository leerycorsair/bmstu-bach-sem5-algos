\chapter{Исследовательская часть}

\section{Пример работы}

Демонстрация работы программы приведена на рисунке \ref{img:demo}.

\boximg{180mm}{demo}{Демонстрация работы алгоритма поворота}

\section{Технические характеристики}

Технические характеристики устройства, на котором выполнялось тестирование:

\begin{itemize}
	\item Операционная система: Windows 10 64-bit \cite{windows}.
	\item Память: 16 GB.
	\item Процессор: AMD Ryzen 5 4600H \cite{amd} @ 3.00 GHz.
\end{itemize}

Тестирование проводилось на ноутбуке при включённом режиме производительности. Во время тестирования ноутбук был нагружен только системными процессами.

\section{Время выполнения алгоритмов}
 В таблице \ref{tabular:full_hd} и на рисунке \ref{plt:time_full_hd} приведены зависимости времени работы алгоритмов для различного количества потоков (количество точек соответствует разрешению 1920х1080).
 
 \captionsetup{singlelinecheck = false, justification=raggedright}
 \begin{table}[h!]
 	\begin{center}
 		\caption{\label{tabular:full_hd}Результат времени работы алгоритма при различном количестве потоков}
 		\begin{tabular}{|c|c|}
 			\hline
 			Количество потоков & Время, тики \\ 
 			\hline
 			1 & 1039794938  \\\hline
 			2 & 738458153  \\\hline
 			4 & 380418539  \\\hline
 			8 & 139540901  \\\hline
 			16 & 95889809  \\\hline
 			32 & 67919394  \\\hline
 		\end{tabular}
 	\end{center}
 \end{table}
 \captionsetup{singlelinecheck = false, justification=centering}

\begin{figure}[!h]
	\centering
	\begin{tikzpicture}
		\begin{axis}[
			axis lines=left,
			xlabel=Количество потоков,
			ylabel={Время, тики процессора},
			legend pos=north west,
			ymajorgrids=true
			]
			\addplot table[x=thrds,y=ticks,col sep=comma] {inc/csv/full_hd.csv};
			\legend{Full HD (1920x1080)}
		\end{axis}
	\end{tikzpicture}
	\captionsetup{justification=centering}
	\caption{Сравнение алгоритмов на различных количествах потоков}
	\label{plt:time_full_hd}
\end{figure}

В таблице \ref{tabular:points_cmp} и на рисунке \ref{plt:time_points_cmp} приведены зависимости времени работы алгоритмов для различного количества точек.

 \captionsetup{singlelinecheck = false, justification=raggedright}
\begin{table}[h!]
	\begin{center}
		\caption{\label{tabular:points_cmp}Результат времени работы алгоритма при различном количестве точек}
		\begin{tabular}{|c|c|c|}
			\hline
			Количество точек & 1 поток & 12 потоков\\ 
			\hline
			4000&941572&1818340   \\\hline
			16000&2733740&1960482 \\\hline
			64000&10062319&3676082 \\\hline
			256000&38439025&7296946 \\\hline
			512000&76648487&10203703 \\\hline
			2048000&303856737&29792272 \\\hline
		\end{tabular}
	\end{center}
\end{table}
\captionsetup{singlelinecheck = false, justification=centering}

\begin{figure}[!h]
	\centering
	\begin{tikzpicture}
		\begin{axis}[
			axis lines=left,
			xlabel=Количество потоков,
			ylabel={Время, тики процессора},
			legend pos=north west,
			ymajorgrids=true
			]
			\addplot table[x=points,y=ticks,col sep=comma] {inc/csv/single.csv};
			\addplot table[x=points,y=ticks,col sep=comma] {inc/csv/multi.csv};
			\legend{1 поток, 12 потоков}
		\end{axis}
	\end{tikzpicture}
	\captionsetup{justification=centering}
	\caption{Сравнение алгоритмов при различном количестве элементов}
	\label{plt:time_points_cmp}
\end{figure}

\captionsetup{singlelinecheck = false, justification=centering}


\section{Вывод}

Исходя из полученных эксперементальных данных можно сделать следующие выводы.

Использование многопоточности однозначно может дать существенный выигрыш по времени работы алгоритма. В алгоритме поворота фигуры двумерного растра использование многопоточности позволило добиться ~15-кратного увеличения производительности.

Следует отметить, что использование многопоточности на небольшом количестве данных может быть неэффективно, т.к. накладные расходы на создание потоков будут превышать выигрыш от параллельных вычислений.


