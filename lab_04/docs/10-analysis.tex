\chapter{Аналитическая часть}

Задача поворота фигуры, представленной в виде массива точек
в двумерном растре, является весьма актуальной, т.к.  компьютерная графика стала неотъемлемой частью повседневной интернет-жизни человека, и существует потребность в быстром рендеринге изображения, например, при анимации поворота фигуры на двумерном растре. Как известно, в экранной плоскости изображение представляет из себя набор пикселей (точек). Очевидно, что какая-либо фигура - это тоже набор точек экранной плоскоти, и для поворота фигуры требуется над каждой ее точкой произвести преобразование для получения новой позиции.

Если использовать один поток для рендеринга изображения, то
при большом количестве и сложности фигур изображение будет генерироваться ощутимо долго, что будет приносить человеку дискомфорт при восприятии, однако если распараллелить этот процесс, т.е. параллельно генерировать части изображения (т.к. эта операция выполняется независимо для каждой точки), то это может дать коллосальной прирост производительности.

\section{Алгоритм поворота точек двумерного растра}

Пусть необходимо повернуть точку $P(x,y)$ вокруг начала координат $O$ на угол $\phi$ \cite{point_rotate}. Изображение новой точки обозначим $P'(x',y')$. Всегда существуют четыре числа $a,b,c,d$ такие, что 
новые координаты могут быть вычислены по значениям старых координат из следующей системы уравнений:
\begin{equation}
	\label{base}
	\begin{cases}
		x'=a\cdot x+b\cdot y
		\\
		y'=c\cdot x+d\cdot y
	\end{cases}
\end{equation}

Для получения значений $a,b,c,d$ рассмотрим точку $P(x,y) = (1,0)$. Полагая $x=1$ и $y=0$ в уравнении \ref{base}, получим:
\begin{equation}
	\label{next}
	\begin{cases}
		x = a
		\\
		y = c
	\end{cases}
\end{equation}

Но в этом простом случае, значения $x'$ и $y'$ равны соответственно $\cos(\phi)$ и $\sin(\phi)$. Тогда имеем:

\begin{equation}
	\label{a_c}
	\begin{cases}
		a = \cos(\phi)
		\\
		c = \sin(\phi)
	\end{cases}
\end{equation}

Аналогичным образом рассматривая точку $P(x, y) = (0, 1)$, получим:

\begin{equation}
	\label{b_d}
	\begin{cases}
		b = -\sin(\phi) 
		\\
		d = \cos(\phi)
	\end{cases}
\end{equation}

Тогда всесто системы уравнений \ref{base} можно записать:
\begin{equation}
	\label{final}
	\begin{cases}
		x' = x\cdot\cos(\phi) - y\cdot\sin(\phi) 
		\\
		y' = x\cdot\sin(\phi) + y\cdot\cos(\phi) 
	\end{cases}
\end{equation}

Система уравнений \ref{final} описывает поворот вокруг точки $O$ - начала системы координат, но часто нужно нужно выполнить поворот относительно заданной точки $(x_c, y_c)$. Тогда система \ref{final} примет следующий вид:
\begin{equation}
	\label{final}
	\begin{cases}
		x' = x_c + (x-x_c)\cdot\cos(\phi) - (y-y_c)\cdot\sin(\phi) 
		\\
		y' = y_c + (x-x_c)\cdot\sin(\phi) + (y-y_c)\cdot\cos(\phi) 
	\end{cases}
\end{equation}

Часто такие преобразования удобно представить в виде матричных преобразований:
\begin{equation}
	\label{mtr}
	M = \begin{pmatrix}
		\cos(\phi) & -\sin(\phi)& 0 \\
		\sin(\phi) & \cos(\phi) & 0 \\
		0 & 0 & 1
	\end{pmatrix}
\end{equation}

Таким образом, распараллеливание будет заключаться в том, что
массив точек будет разбиваться на подмассивы, для каждого из которых независимо от других будет решаться задача преобразования.



\section{Вывод}

В данном разделе была проанализирована предметная область,
установлена актуальность задачи, также был рассмотрен алгоритм задачи, которая будет подвергнута распараллеливанию.


