\chapter*{Введение}
\addcontentsline{toc}{chapter}{Введение}

Многопоточность - это специализированная форма многозадачности, и многозадачность - это функция, которая позволяет вашему компьютеру одновременно запускать две или несколько программ. В общем, существует два типа многозадачности: основанные на процессах и потоки \cite{cpp_concurrency}.

Многозадачность на основе процессов управляет одновременным выполнением программ. Многозадачность на основе потоков связана с одновременным выполнением частей одной и той же программы.

Многопоточная программа содержит две или несколько частей, которые могут запускаться одновременно. Каждая часть такой программы называется потоком, и каждый поток определяет отдельный путь выполнения.


Целью данной лабораторной работы являются изучение и реализация многопоточности на основе алгоритмов компьютерной графики, в частности алгоритма поворота двумерной фигуры, представленной в виде массива точек.

Для достижения указанной выше цели следует выполнить следующие задачи:
\begin{itemize}
	\item изучить понятие параллельных вычислений и алгоритм поворота фигуры в двумерном растре;
	\item привести схемы рассматриваемого алгоритма в последовательном и параллельном вариантах;
	\item реализовать последовательный и параллельный алгоритм поворота фигуры в двумерном растре на одном из языков программирования, используя нативные потоки;
	\item провести сравнительный анализ реализаций алгоритма на основе экспериментальных данных;
	\item описание и обоснование полученных результатов в отчете о выполненной лабораторной работе.
\end{itemize}

