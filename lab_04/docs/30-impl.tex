\chapter{Технологическая часть}

В данном разделе приведены требования к программному обеспечению, средства реализации и листинги кода.

\section{Требования к ПО}

К программе предъявляется ряд требований:
\begin{itemize}
	\item взаимодействие программы и пользователя реализованно в виде меню;
	\item пользователь может ввести фигуру в виде массива точек для поворота ручным способом или сгенерировать, заполнив случайными значениями;
	\item элементами массива могут быть вещественные числа;
	\item считается, что пользователь вводит корректные данные;
	\item на выходе — преобразованный массив точек фигуры двумерного растра;
	\item программа обладает автоматизированной системой сравнительного анализа скорости работы алгоритма при различном количестве потоков.
\end{itemize}

\section{Средства реализации}

В качестве языка программирования для реализации данной лабораторной работы был выбран многопоточный язык С++ \cite{cpp}. Данный выбор обусловлен моим желанием расширить свои знания в области применения данного языка.

\section{Листинг кода}

В листингах \ref{lst:figure}--\ref{lst:utils} приведены реализации алгоритмов поворота, а также вспомогательные функции.



\captionsetup{singlelinecheck = false, justification=raggedright}

\begin{lstinputlisting}[
	caption={Многопоточная функция поворота фигуры},
	label={lst:figure},
	style={cpp},
	linerange={28-51},
	]{../src/figure.cpp}
\end{lstinputlisting}

\begin{lstinputlisting}[
	caption={Функция поворота точки отностельно другой},
	label={lst:point},
	style={cpp},
	linerange={26-33},
	]{../src/point.cpp}
\end{lstinputlisting}

\clearpage

\begin{lstinputlisting}[
	caption={Функция для замера тиков процессора},
	label={lst:utils},
	style={cpp},
	linerange={3-15},
	]{../src/utils.cpp}
\end{lstinputlisting}


В таблице \ref{tabular:func_test} приведены функциональные тесты для алгоритмов сортировки. Все тесты пройдены успешно (таблица \ref{tabular:func_test_res}).


\begin{table}[h!]
	\begin{center}
		\caption{\label{tabular:func_test}Ожидаемый результат работы программы}
		\begin{tabular}{|c|c|c|}
			\hline
			Входной массив & Угол поворота & Ожидаемый результат \\ 
			\hline
			$[[1,2], [3,4], [5,6]]$ & 45 & $[[-0.70, 2.12], [-0.70, 4.94], [-0.70, 7.77]]$\\\hline
			$[[1,2], [3,4], [5,6]]$ & 360 & $[[1,2], [3,4], [5,6]]$\\\hline
			$[[1,2], [3,4], [5,6]]$ & 180 & $[[-1,-2], [-3,-4], [-5,-6]]$\\\hline
			$[[1,2], [3,4], [5,6]]$ & -90 & $[[2,-1], [4,-3], [6,-5]]$\\\hline
		\end{tabular}
	\end{center}
\end{table}

\begin{table}[h!]
	\begin{center}
		\caption{\label{tabular:func_test_res}Фактический результат работы программы}
		\begin{tabular}{|c|c|c|}
			\hline
			Входной массив & Угол поворота & Фактический результат \\ 
			\hline
			$[[1,2], [3,4], [5,6]]$ & 45 & $[[-0.70, 2.12], [-0.70, 4.94], [-0.70, 7.77]]$\\\hline
			$[[1,2], [3,4], [5,6]]$ & 360 & $[[1,2], [3,4], [5,6]]$\\\hline
			$[[1,2], [3,4], [5,6]]$ & 180 & $[[-1,-2], [-3,-4], [-5,-6]]$\\\hline
			$[[1,2], [3,4], [5,6]]$ & -90 & $[[2,-1], [4,-3], [6,-5]]$\\\hline
		\end{tabular}
	\end{center}
\end{table}


\captionsetup{singlelinecheck = false, justification=centering}

\section{Вывод}

В данном разделе был реализован и протестирован спроектированный алгоритм поворота точек фигуры в двумерном растре.
