\chapter*{Заключение}
\addcontentsline{toc}{chapter}{Заключение}

В ходе выполнения работы были выполнены все поставленные задачи и изучены методы динамического программирования с реализацей многопоточности на основе алгоритма поворота фигуры двумерного растра.

Использование многопоточности однозначно может дать существенный выигрыш по времени работы алгоритма. В алгоритме поворота фигуры двумерного растра использование многопоточности позволило добиться ~15-кратного увеличения производительности.

Следует отметить, что использование многопоточности на небольшом количестве данных может быть неэффективно, т.к. накладные расходы на создание потоков будут превышать выигрыш от параллельных вычислений.

